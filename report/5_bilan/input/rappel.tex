\section{Rappel des objectifs}
\label{sec:rappel}

\subsection{Notre problématique}
	L’accès aux documents anciens à l’ère du numérique est une problématique réelle. Plusieurs millions de journaux datant de la révolution française jusqu’à aujourd’hui sont actuellement présents dans les archives. De manière générale, l’accès aux documents d’archives est améliorable. En effet, celui-ci est restreint pour des raisons de conservation du document. De plus, certaines archives contiennent des kilométrages de rayons et, même si dans le cadre de la presse ancienne le nombre de documents stockés est moindre, il n’en reste que l'accès à des documents traitant d'un sujet précis en est d'autant plus complexe. Cependant la numérisation ne résoud pas entièrement le problème de l’accessibilité. Il est aussi nécessaire de rendre ces documents accessibles par internet et accompagné d'outils rendant cet accès simplifié.

\subsection{Nos objectifs}

	Les objectifs de notre application tournent donc autour de trois grands thèmes :
	
	\textbf{Rechercher :} L'utilisateur doit pouvoir obtenir facilement des articles traitant d'un sujet particulier. Cela signifie notamment un recherche plein texte dans le contenu de l'article. La recherche doit se faire rapidement malgré la grande quantité d'articles contenus dans la base de données. Ensuite, l'utilisateur aura la possibilité d'affiner sa recherche à l'aide d'outils de tri et de filtrage. Ce filtrage peut concerner les dates, les journaux ou les tags associés aux articles. Enfin, l'utilisateur pourra aussi simplement trouver l'édition particulière d'un journal.
	
	\textbf{Visionner :} La visualisation des journaux doit être appropriée à des grandes images et faciliter la lecture. De plus, des outils seront proposés à l'utilisateur afin de l'aider à naviguer sur la page et ainsi trouver les informations recherchées. Cela peut se faire, par exemple, à l'aide d'une recherche de mots-clés sur les page. De même, les informations sur le journal et l'article doivent être affichées. Enfin, des articles proches seront aussi proposés à l'utilisateur afin de l'aider à approfondir sa recherche.

	\textbf{Contribuer :} Il est, de même, essentiel de mettre en avant la richesse de la presse ancienne. Notre plateforme propose donc le système de revue de presse. C'est-à-dire un ensemble d'articles ayant un thème en commun que les utilisateurs veulent mettre en avant. Chaque utilisateur pourra contribuer à ces revues. Il sera d'ailleurs possible de faire une recherche parmi les revues de presse. Les tags assignés aux articles sont donnés par les utilisateurs. Lors de l'ajout d'un tag, on doit avoir accès aux tags déjà existant par autocomplétion. Enfin, la plateforme proposera un système de compte utilisateur. Ainsi, les revues de presse seront reliées au compte de leur créateur. Les utilisateurs pourront aussi stocker les revues de presse auxquelles ils ont ajouté des articles ainsi qu'une liste de leurs articles favoris.