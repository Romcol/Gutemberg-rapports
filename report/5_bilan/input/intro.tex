\section{Introduction}
\label{sec:intro}

Le projet, inscrit dans le cadre de la formation de 4ème année du département Informatique de l’INSA de Rennes et proposé par l’équipe Intuidoc de l’Irisa, a pour but d’améliorer l’accès aux archives de presse locale et ancienne. L’accès aux documents anciens à l’ère du numérique est une problématique réelle. Notre projet concerne donc la mise à disposition de documents de presse ancienne au travers d’une plateforme accessible à tous. C’est à partir d’un prototype, conçu par l’équipe Intuidoc, que nous créerons cette plateforme.
Ce prototype permet d’interpréter des images de journaux anciens (XIXe et XXe) afin de produire une représentation XML du contenu de chaque page. Par ailleurs, Yoann Royer, chef de projet chez Sopra Steria et ancien diplômé de l’INSA, nous offre son accompagnement et ses conseils tout au long du projet.

Le projet se déroule en deux phases : la phase d'analyse et la phase de conception. La première phase qui fut réalisée au premier semestre, consistait à réaliser une pré-étude, une spécification générale, ainsi qu'une planification  du projet. La seconde phase, que nous avons commencé au second semestre concerne la conception et l'implémentation de l'application. 

Pour la première phase, nous étions sept étudiants à travailler sur la phase d'analyse. À partir du second semestre, nous étions trois étudiants pour la conception et l'implémentation de l'application.

Ce rapport fait le bilan du projet. Cela comprend les modifications apportées aux rapports de conception\cite{Conc} et de spécification\cite{Specs} qui ont été réalisées au cours du développement de notre application. De même, ce bilan se fera au travers des différentes étapes du développement du projet que nous détaillerons du point de vue de la planification et des tâches réalisées.
