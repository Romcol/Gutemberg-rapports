\section{Conclusion}
\label{sec:conc}

Ce rapport a permis de mettre en évidence et de décrire en détail la routine d'un utilisateur et toutes les fonctionnalités auquel il pourra accéder. On retrouve donc la recherche d'éléments suivant certains paramètres comme des tags, des dates ou des noms de journaux qui est une fonctionnalité importante et devra être le plus efficace possible. La consultation de documents permettant de naviguer à travers des journaux ou des revues de presse, de sélectionner des articles pour les ajouter à une revue de presse ou voir des articles associés par exemple est aussi une fonctionnalité primordiale. La gestion d'un compte utilisateur, ou d'autres fonctionnalités "mineures", qui sembleraient anodines, ajouterons une valeur à la plateforme qui va la différencier des autres. La définition de toutes ces fonctionnalités nous a donc permis d'établir une liste des tâches, plus ou moins précise, pour imaginer une première planification générale pour la suite. 

La prochaine étape du projet est la planification initiale du développement. Dans ce rapport nous avons établi une simple planification globale du projet à partir des fonctionalités définies et détaillés. L'intérêt maintenant est de découper plus en détail les étapes du développement, l'organisation de celui-ci, mais surtout une évaluation précise et plus exacte des ressources qui seront nécessaires pour chaque tâche. Nous nous avancerons aussi sur la phase de conception du projet, et ce afin de pouvoir démarrer au plus tôt le développement de l'application pour ne pas être limité pas nos ressources.
