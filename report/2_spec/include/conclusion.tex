\section{Conclusion}
\label{sec:conc}

Ce rapport a permis de mettre en évidence et de décrire en détail la routine d'un utilisateur et toutes les fonctionnalités qui lui seront accessibles. On retrouve donc la recherche d'éléments suivants certains paramètres comme des tags, des dates ou des noms de journaux ; qui est une fonctionnalité importante et devra être le plus efficace possible. La consultation de documents permettant de naviguer à travers des journaux ou des revues de presse, de sélectionner des articles pour les ajouter à une revue de presse ou de voir des articles associés, par exemple, est aussi une fonctionnalité primordiale. La gestion d'un compte utilisateur, ou d'autres fonctionnalités "mineures", qui sembleraient anodines, ajouterons une valeur à la plateforme qui va la différencier des autres. Définir ces fonctionnalités nous a permis d'établir une liste des tâches qui reste assez globale, afin d'imaginer une première planification générale pour la suite. Ce planing sera détaillé ultérieurement.

La prochaine étape du projet est la planification initiale du développement. Dans ce rapport nous avons établi une simple planification globale du projet à partir des fonctionnalités définies et détaillées. L'intérêt est maintenant de découper plus en détail les étapes du développement, l'organisation de celui-ci, mais surtout de réaliser une évaluation précise de chaque tâche et donc une évaluation plus exacte des ressources qui seront nécessaires pour chaque tâche. Nous nous avancerons aussi sur la phase de conception du projet et ce afin de pouvoir démarrer au plus tôt le développement de l'application pour ne pas être limité pas nos ressources. Cela nous permettra d'avoir une marge de temps et d'erreur pour le développement plus grande.