\section{Recherche d'éléments}
\label{sec:recherche}

On va creer un web site qui joue le role comme un "moteur de recherche". Dans ce site, il y aura certains champs pour qu'on puisse chercher des articles qu'on veut. Au minimal, on va definir un champ ou l'utilisateur peut entrer des mots cles pour chercher un article. 

En gros, on a trois types de resultat que l’on veut chercher: par Journal, par Article et par Revue-presse. Ce sont des trois principaux types de resultat du recherche que l’on vous propose. L’utilisateur ne peut pas chercher ces trois types en meme temps. Chaque recherche est effectuee  par un seul type de resultat du recherche. Si l’utilisateur veut chercher un autre type. Il y a un bouton pour retourner vers la page d’acceuille de recherche et l’utilisateur peut faire une autre recherche sur l’autre type. La raison pour laquelle on ne fait pas de melange entre les types de resultat est de distinguer d’abord les types et en plus, il ne va pas rendre trop de resutats pour ne pas perturber l’utilisateur. Car le nombre des articles de chaque type de plus en plus devient enorme, en plus il y a trois types de resultat dans notre base de donnee, alors les resultats retournes peuvent perturber les utilisateurs en ne mettant pas le bon type de recherche sur la premiere page. Alors, pour assurer que les utilisateurs trouveront de bons types de recherche de maniere efficace, on ne melange pas ces trois types de recherche.

En haut de chaque page, il y a des boutons qui permettent les utilisateurs de choisir quelle facon de recherche et des filtrages possibles pour que l’on peut obtenir des resultats de recherche le plus convenable et le plus proche de ce qu’on expecte.

D’abord, on propose des facons de recherches: 

\begin{itemize}
\item Exact: L'utilisateur peut trouver exactement les mots qu'il veut chercher dans toutes les articles dans la base de donnee. Par exemple: Si l'utilisateur cherche "Rennes", il va forcement trouver toutes les articles qui contiennent le mot Rennes dedans. Cette facon est tres utile pour des mots cles que l’on connait exactement dans le context. Elle va nous donner moins de resultats qui nous perturbe moins et nous renvoie exactement ce que l’on veut chercher.
\item Contenu: Les mots que l'utilisateur peut etre une partie d'une autre mot dans le context. Alors ce type de recherche va nous retourner plus de resultat mais il y a aussi des resultats ne sont pas relies a ce que l'utilisateur veut chercher. Par exemple: Si l'utilisateur veux chercher "Re", les resultats possibles sont "repartition", "Rennes",… Cette facon nous donne plus de resultats et aussi tres utile si l’on ne connait pas le mot exact, alors l’utilisateur peut entre quelque caracteres d’un mot et notre moteur de recherche va lui renvoyer toutes les reponses dont ses contenues sont associes a l’ensemble des carateres cherches. Mais comme dans l’exemple "Re" qui donne "repartition", "Rennes" que j’ai indique ci-dessus. Cette methode a aussi son point negative.tres facile a voir dans l’exemple. Elle peut retourner des resultats qui ne relient pas du tout a ce que l’on veut chercher exactement et nous perturbe avec un grand nombre de resultat de retourne.
\item Multi-recherche: Des fois, si l’utilisateur veut faire plusieurs de recherches, il doit faire une recherche, retourner sur la page accueil (optionnel), ensuite lancer une autre recherche qui perd plus de temps et des fois gene des utilisateurs. Donc, au lieu de chercher chaque fois un seul mot ou une seule phrase, la facon multi-recherche nous permet de faire plusieurs recherches en meme temps. L’utilisateur peut entrer plusieurs mots et plusieurs phrases et lancer plusieurs recherches en meme temps. Les mots et les phrases dans chaque recherche seront delimites par une virgule. La liste de resultat sera l’ensemble des resultats de chaque recherche. Cette facon nous permet de gagner plus de temps en recherchant plusieurs fois en meme temps. En gros, elle est la version multi-exact, car elle va retourner l’ensemble de resultats par la recherche exacte.
\end{itemize}

	Ensuite, on propose le filtrage pour les recherches. Les informations d’un article sont divisees dans le tableau de la base de donnee comme: title, contenu de l’article, date sortie, auteur, ... Un jour, si l’utilisateur n’arrive pas a se souvenir de sujet ou les mots dans un article, or si il veut chercher tous les articles d’un auteur a une date fixee. C’est la raison pour laquelle on met ce filtrage dans notre moteur de recherche. On peut dire que c’est une facon de recherche aussi. Du coup, il y a autant de filtrage que les informations necessaires a stocker pour un article. Donc, les utilisateurs peuvent aussi chercher des articles par le title, des mots cles dans son contenu, par sa date ou par son auteur, ...
 
Si l'utilisateur cherche plusieurs mots ou une phrase entiere. On va separer chaque mot dans la phrase, ensuite on cherche tous les mots possibles dans la base de donnee et on retourne le resultat. Par exemple: Si l'utilisateur cherche "Rennes est une ville", notre moteur de recherche va chercher dans la base de donnee toutes les articles qui contiennent chaque mot separe : "Rennes", "est", "une", "ville". Tous ces mots separes seront cherches dans chaque colonne du tableau dans la base de donnee. Quand il arrive a le trouver, il va retourner le record entier qui contient le mot trouve. 

On va creer aussi un site web qui genere automatiquement les resultats qu'on recoit par la requette envoye a la base de donnee. En generant le site web de resultat, on va souligner ou mettre ces mots cherches en font bold.  