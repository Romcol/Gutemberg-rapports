\section{Recherche d'éléments}
\label{sec:recherche}

La plateforme joue en quelque sorte le rôle d'un moteur de recherche. L'idée est de fournir aux utilisateurs une recherche riche et diverse qui lui permettra de trouver efficacement des journaux, des articles ou des revues de presse qu'il souhaiterait consulter. On veut donc permettre à l'utilisateur d'effectuer trois types de recherches; suivant le type des documents qu'il recherche. Un utilisateur ne pourra pas chercher en même temps des 'articles' et des 'journaux', ce afin de distinguer ces différents types et d'éviter de rendre trop de résultats inintéréssants. Suivant le type de document recherché, l'utilisateur disposera de plusieurs filtres.

\textbf{Cette partie était bancale, une réécriture est nécessaire après discussion et nouvel accord sur les fonctionnalités}

\subsection{Recherche de journaux}
\label{sec:recherche_journal}

filtre dispo : date, titre (une)

\subsection{Recherche d'articles}
\label{sec:recherche_article}

filtre dispo : date, journal, titre, tags, recherche plein texte

\subsection{Recherche de revues}
\label{sec:recherche_revue}

filtre dispo : nom



    \begin{figure}[H]
        \centering
        \includegraphics[width=\textwidth]{figures/manquant.png}
            \caption{Schéma manquant}
            \label{fig:manquant}
    \end{figure}