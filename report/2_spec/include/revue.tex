\section{Les revues de presse}
\label{sec:revue}

La principale fonctionnalité innovante de notre plateforme de consultation de journaux anciens est la possibilité de
créer et de partager des revues de presse.

Une revue de presse est, basiquement, une liste d'articles de journaux regroupés ensemble par un utilisateur autour d'une thématique précise,
comme par exemple une date commune, un événement commun ou une zone géographique.

La création des revues de presse pourra se faire à deux moments distincts. En cliquant sur un bouton depuis le menu principal, l'utilisateur se verra redirigé vers une page de création de revue de presse. Cette page contiendra des champs pour nommer la revue de presse et lui mettre un commentaire descriptif. L'utilisateur pourra ensuite ajouter des articles à sa revue de presse et pourra la consulter depuis son profil ou depuis le moteur de recherche. La deuxième manière d'accéder à la page de création d'une revue de presse sera depuis la page de visualisation, lors du clique sur le bouton d'ajout d'un document à une revue de presse. Il s'agit de la même page décrite précédemment,
la seule différence étant que la revue de presse sera initialisée avec le-dit document.

Il y aura plusieurs moyens d'ajouter un document à une revue de presse :  

\begin{itemize}
  \item cliquer sur le bouton d'ajout présent sur la page de visualisation et choisir la
revue de presse à laquelle contribuer.
  \item cliquer sur le bouton d'ajout directement présent depuis le moteur de recherche et choisir
à quelle revue de presse contribuer.
  \item cliquer depuis la page récapitulative d'une revue de presse sur le s boutons d'ajout de chaque document.
\end{itemize}
Depuis la page récapitulative d'une revue de presse, il sera aussi possible, pour le propriétaire de la revue de presse, de retirer un article de
la revue de presse.

La page de visualisation d'une revue de presse sera composé du titre, de la description de la revue de presse et d'une liste comprenant les articles
de cette revue de presse. Depuis cette page il sera possible de modifier le titre, la description et de retirer un article (si on en est le créateur) de cette revue de presse.