\section{Les revues de presse}
\label{sec:revue}

La principale fonctionnalité innovante de notre plateforme de consultation de journaux anciens est la possibilité de
créer et de partager des revues de presse.

Une revue de presse est, basiquement, une liste d'articles de journaux regroupés ensemble par un ou plusieurs utilisateurs autour d'une thématique précise,
comme par exemple une date, un événement ou une zone géographique.

la seule différence étant que la revue de presse sera initialisée avec le-dit document.

Il y aura plusieurs moyens d'ajouter un document à une revue de presse :

\begin{itemize}
  \item cliquer sur le bouton d'ajout présent sur la page de visualisation et choisir la
revue de presse à laquelle contribuer.
  \item cliquer sur le bouton d'ajout directement présent depuis le moteur de recherche et choisir
à quelle revue de presse contribuer.
  \item cliquer depuis la page récapitulative d'une revue de presse sur les boutons d'ajout de chaque document.
\end{itemize}
Depuis la page récapitulative d'une revue de presse, il sera aussi possible, pour le créateur de la revue de presse, de retirer un article de
la revue de presse.

La page de visualisation d'une revue de presse serae composé du titre, de la description de la revue de presse et d'une liste comprenant les articles
de cette revue de presse. Depuis cette page il sera possible de modifier le titre, la description et de retirer un article sur cette revue de presse.

\begin{figure}[H]
    \centering
    \includegraphics[width=\textwidth]{figures/revue.png}
    \caption{Page de visualisation d'une revue de presse}
    \label{fig:revue}
\end{figure}

