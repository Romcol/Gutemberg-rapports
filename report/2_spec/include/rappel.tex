\section{Rappel du projet}
\label{sec:rapp}

Le projet, inscrit dans le cadre de la formation de 4ème année du département Informatique de l’INSA de Rennes et proposé par l’équipe Intuidoc de l’Irisa a pour but d’améliorer l’accès aux archives de presse locale et ancienne. C’est en coopération avec les Archives départementales d’Ile-et-Vilaine que nous obtiendrons un panel de document autour duquel nous pourrons batîr le système de consultation. Ainsi, nous devons mettre à disposition du public une plateforme répondant à cette problématique.
C’est à partir d’un prototype conçu par l’équipe Intuidoc et permettant d’interpréter des images de journaux anciens (XIXe et XXe) afin de produit une représation XML du contenu de chaque page que nous crérons cette plateforme.
Egalement, Yoann Royer, chef de projet chez Sopra Steria et anciennement diplômé de l’INSA nous offre son accompagnement tout au long du projet. 

	\subsection{Des objectifs aux spécifications}
	\label{sec:objectifs}
	En se basant sur les objectifs mentionnés dans le rapport de préetude nous pouvons construire une liste de spécifications aux niveaux de criticités variés.
	

	\textbf{Lire un journal :} cette fonctionalité est le coeur du projet, on souhaîte que l’utilisateur puisse naviguer dans la revue aisément. Il faut donc que notre IHM (Interface Homme Machine) soit adaptée à notre cible. Une large partie de notre publique étant les personnes agées (recherche sur la généalogique par exemple) l’interface doit être suffisamment simple, user-friendly et graphiquement lisible pour l’utilisateur. De même elle se doit d’être adaptée à toutes platforme (PC, tablette, smartphone). Afin d’optimiser la fluidité de l’application nous avons remarqué qu’un tuilage des revues avec chargement asynchrone est nécessaire. (On parle ici d’OpenSeaDragon ?)
	

	\textbf{Recherche avancée :} nous souhaitons que l’application dispose d’une recherche sur les revues prenant en compte un tas de paramètre tel que la date, le sujet, le nom de la revue, le thème etc… Cette base ayant pour vocation de contenir un certain nombre de documents, au départ des milliers puis quelques centaines de milliers voir quelques millions. Cela induit donc l’utilisation d’une base de donnée performante pour la recherche.
	

	\textbf{Recherche plein texte :} la visionneuse se doit d’avoir un outil de recherche plein texte. L’utilisateur se doit d’avoir la possibilité d’effectuer une recherche à partir d’un mot présent dans un article ou un ensemble de mot de même racine ou phonétiquement similaire. La visionneuse doit fournir un ensemble d’article et de passage répondant à la requete et doit mettre les mots correspondant à la recherche en surbrillance. L’outil de recherche doit être adaptée à un document structuré tel que de l’xml (le contenu d’une page étant stocké dans ce format) et fourni un résultat quasi-instantanné. (ElasticSearch ?)
	

	\textbf{Découper et identifier une page de journal :} la visionneuse se doit par un système d’overlays structurer le contenu d’une page selon des articles images ou tout autre élement interessant. La détection du contenu est réalisé par un outil de l’équipe Intuidoc qui fournir un fichier XML contenant le découpage de la page du journal. La visionneuse doit permettre la définition de zones résistantes aux zoom ou autres rotations de l’image et sous forme de polygones. En effet, les zones de la page peuvent être de toute forme.


	\textbf{Proposition d’articles proches :} nous pensons integrer une tel fonctionnalité dans l’outil de visionnage. Tout d’abord en affichant des suggestions par rapport à l’article lu en cours et en proposant par exemple des articles sur le même sujet, les éditions suivant ou précedentes du même journal. De plus, nous souhaitons proposer la créations et la visualisations de parcours de lecture dont on peut rapporter la structure à celle d’une playlist musicale. Il faut ainsi définir une certain nombre de critères de rapprochement et de tri des articles mais aussi un outil permettant de faire cette sélection de façon rapide.
	

	\textbf{Parcours thématiques et système collaboratif :} afin de développer ce système de parcours de lecture. L’application doit bénéficier d’un système de gestion d’utilisateur. Les utilisateurs étant liés à un nombre de parcours qu’ils ont crée. Et inversement un parcours doit pouvoir être la propriété de plusieurs utilisateurs. Les fonctionnalités proposés sont l’ajout, la modification ou la suppression des élements d’un parcours. Idéalement, un tel espace d’entraide devra permettre aux utilisateurs de proposer des corrections sur le contenu plein texte des articles par exemple.