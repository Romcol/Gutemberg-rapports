\section{Organisation du projet}
\label{sec:orga}

	Les différents délivrables du projet 4INFO imposent une gestion de projet suivant un modèle de cycle en V. Cependant, nous sommes libres d'adapter nos propre méthodes de développement 

	Ici, on a de nombreuses fonctionnalités que l'on va être amené à réaliser. Il est possible que l'on ne soit pas capable de toutes les réaliser; nos ressources étant limitées (3 développeurs) tout comme le temps disponible. Il est donc nécessaire de penser à une bonne méthode de gestion lors du développement. D'un côté, il est nécessaire d'évaluer l'importance de chaque fonctionnalité, définie bien précisément, et de l'autre, partie la plus alératoire, il nous faut estimer la durée de délivrable de celles-ci.

	En gardant un modèle de cycle en V, on se retrouverait donc à développer la plateforme, puis à effectuer différentes séries de tests avant de le délivrer. Le problème de cette méthode est que l'on va devoir faire une sélection de fonctionnalités à développer, s'y tenir, et un livrable ne sera fournit qu'à la fin. Entre autre, on ne pourra avoir un retour sur l'application lorsque le développement sera fini. Il sera impossible d'inférer sur les fonctionnalités puisqu'il n'y aura aucun retour durant le temps de développement. 

	C'est pourquoi nous pensons planifier le développement de l'application en se basant sur les méthodes agiles. On aura donc une liste de fonctionnalités, triées suivant leur importance et leur difficulté, à développer. En considérant les ressources mises à disposition, on délivrera et on effectuera en continue des tests pour chacun des fonctionnalités. Ainsi il sera possible de présenter l'avancement au milieu du projet à nos encadrant pour pouvoir adapter les fonctionnalités restantes, ou en rajouter par la suite. Si celles-ci se trouvent être plus intéressantes (plus intéressante/plus facile), on pourrait se retrouver à les développer avant des fonctionnalités définies avant. Enfin, cela permet à un instant T d'avoir un produit fonctionnel.