\section{Introduction}
\label{sec:intro}

<<<<<<< HEAD
Le projet, inscrit dans le cadre de la formation de 4ème année du département Informatique de l’INSA de Rennes, et proposé par l’équipe
Intuidoc de l’Irisa, a pour but d’améliorer l’accès aux archives de presse ancienne. Nous travaillons en partenariat avec les Archives départementales d’Ille-et-Vilaine. L'accès aux documents anciens à l'ère du numérique est une problématique réelle, dont ils s'occupent maintenant depuis plusieurs années. Notre projet concerne donc la mise à disposition de documents de presse ancienne au travers d'une plateforme accessible à tous. C'est à partir d'un prototype conçu par l'équipe Intuidoc et permettant d'interpréter des images de journaux anciens (XIXe et XXe) afin de produire une représentation XML du contenu de chaque page, que nous créerons cette plateforme. Par ailleurs, Yoann Royer, chef de projet chez Sopra Steria, et ancien diplômé de l'INSA, nous offre son accompagnement et ses conseils tout au long du projet.

Le précédent rapport avait fait état des objectifs du projet et de l'existant autour de celui-ci. Une première définition des fonctionnalités, brèves, avait été établie, afin de réfléchir aux technologies qui nous permettraient de développer la plateforme et mettre en place celles-ci. Ce rapport de spécifications fonctionnelles va nous permettre d'établir la liste quasi-exhaustive des spécifications fonctionnelles tout en les détaillant au maximum. Des croquis seront aussi fournis pour illustrer nos propos et donner une meilleure vue d'ensemble de la plateforme.

    \subsection{Objectifs généraux de la plateforme}
    \label{sec:objectifs}
    En se basant sur les objectifs mentionnés dans le rapport de préetude\footnote{Rapport de pré-étude, Projet Gutenberg 4INFO 2015-2016}, nous avons établi une liste de fonctionnalités aux niveaux de criticités variées qui seront donc développés dans ce rapport.
=======
Le projet, inscrit dans le cadre de la formation de 4ème année du département Informatique de l’INSA de Rennes et proposé par l’équipe Intuidoc de l’Irisa, a pour but d’améliorer l’accès aux archives de presse locale et ancienne. Nous travaillons d'ailleurs en partenariat avec les Archives départementales d'Ille-et-Vilaine. L'accès aux documents anciens à l'ère du numérique est une problématique réelle, dont ils s'occupent maintenant depuis plusieurs années. Notre projet concerne donc la mise à disposition de documents de presse ancienne au travers d'une plateforme accessible à tous. C'est à partir d'un prototype conçu par l'équipe Intuidoc, que nous créerons cette plateforme. Ce prototype permet d'interpréter des images de journaux anciens (XIXe et XXe) afin de produire une représentation XML du contenu de chaque page. Par ailleurs, Yoann Royer, chef de projet chez Sopra Steria et ancien diplômé de l'INSA, nous offre son accompagnement et ses conseils tout au long du projet.

Le précédent rapport avait fait état des objectifs du projet et de l'existant autour de celui-ci. Une première définition des fonctionnalités, brève, avait été établie, afin de réfléchir aux technologies qui nous permettraient de développer la plateforme et de mettre celle-ci en place. Ce rapport de spécifications fonctionnelles va nous permettre d'établir la liste quasi-exhaustive des spécifications fonctionnelles tout en les détaillant au maximum. Des croquis seront aussi fournis pour imager nos propos et donner une meilleure vue d'ensemble de la plateforme.

    \subsection{De la problématique aux objectifs}
		\label{sec:probleme}
		
		L'accès à la presse ancienne est une tâche compliquée. Il faut se rendre dans des archives pour pouvoir accéder au document. Certains étant vieux et abîmés, ils ne sont pas forcément accessibles au public. L'utilisation d'une plateforme web est une solution toute adéquate pour rendre plus facile et plus simple la consultation de ces documents. Et cela permet de conserver le document original. Nous avons constaté grâce à notre recherche de l'existant qu'il existe déjà plusieurs sites web qui proposent de consulter des documents anciens tels que des journaux mais aussi des papiers administratifs, des lettres, des registres paroissiaux... 
		
		Cependant, cette facilitation de consultation amène un autre problème : l'accès à l'information, et c'est une problématique conséquente. En effet, la presse ancienne comprend un nombre énorme de documents car ce sont des journaux, quotidien pour certains, et sur plusieurs dizaines d'années. Cela n'est pas gênant si la personne ne souhaite que feuilleter parmi les documents, mais cela est plus problématique si elle veut trouver un article qui parle d'un sujet particulier ou si elle les journaux d'une date particulière. Il est donc essentiel de proposer des outils de recherche afin de trouver l'information désirée dans la masse de documents.
			
		Enfin, il est important de valoriser les documents qui sont mis à disposition. En effet, ils représentent l'histoire de la France et du monde. Il faut donc mettre en avant la richesse de la presse ancienne auprès des personnes qui consultent les documents. Cela peut se faire en les guidant vers d'autres articles qui portent le même sujet que celui qu'elles consultent. Cela peut être d'ailleurs fait par les personnes elles-mêmes qui seraient passionnées par un sujet particulier et qui voudraient le faire découvrir d'autres personnes.
		
		\subsection{Des objectifs aux spécifications}
    \label{sec:objectifs}
    En se basant sur les objectifs mentionnés dans le rapport de préetude\footnote{Rapport de pré-étude, Projet Gutemberg 4INFO 2015-2016}, nous avons établi une liste de fonctionnalités aux niveaux de criticités variées qui seront donc développées dans ce rapport.
>>>>>>> 1332c7553389d869a280204ebb3a682f6c01895e


    \textbf{Lire un journal :} cette fonctionnalité est le cœur du projet ; l'utilisateur doit pouvoir naviguer dans la revue aisément.
    Notre IHM (Interface Homme Machine) sera donc adaptée à notre cible. L’interface sera suffisamment simple et graphiquement lisible pour l'utilisateur. De même, elle sera adaptée à toute plateforme (PC, tablette, smartphone). Afin d’optimiser la fluidité de l’application, nous avons remarqué
    qu’un tuilage des revues avec chargement asynchrone est nécessaire.


    \textbf{Recherche avancée :} l’application disposera d’une recherche sur les revues prenant en compte plusieurs paramètres
    tels que la date, le sujet, le nom de la revue, le thème, etc. Cette base aura pour vocation de contenir un certain nombre de documents, au
    départ quelques milliers puis plusieurs centaines de milliers voir quelques millions. Cela induit donc l’utilisation d’une base de données
    performante pour la recherche.

<<<<<<< HEAD
    \textbf{Recherche dans la page :} la visionneuse bénéficiera d'un outil de recherche plein texte. L'utilisateur aura la possibilité
    d’effectuer une recherche à partir d’un mot présent dans un article ou un ensemble de mots de même racine ou phonétiquement similaires.
    La visionneuse doit fournir un ensemble d’articles et de passages répondant à la requête et doit mettre en avant les mots correspondants à la recherche pour facilement les repérer.
    L’outil de recherche doit être adapté à un document structuré tel que de l’XML (le contenu d’une page étant stocké dans ce format) et fourni
    un résultat quasi-instantané (< 2 secondes).


    \textbf{Découper et identifier une page de journal :} la visionneuse possèdera un système de calques transparent qui permettra de structurer le contenu d’une page selon
    les articles, images ou tout autre élément intéressant. La détection du contenu est réalisée par un outil de l’équipe Intuidoc qui fournit un
    fichier XML contenant le découpage de la page du journal. La visionneuse doit permettre la définition de zones résistantes au zoom ou autres
    rotations et les calques transparents doivent pouvoir prendre la forme de polygones non convexes. En effet, les articles peuvent avoir des formes complexes.


    \textbf{Proposition d’articles proches :} nous allons intégrer une telle fonctionnalité dans l’outil de visionnage. L'utilisateur se verra
    suggérer des articles en rapport à celui en cours de lecture ; par exemple des articles sur le même sujet ou suivant les éditions suivantes ou précédentes
    du même journal. De plus, nous souhaitons proposer la création et la visualisation de parcours de lecture dont on peut rapporter la structure
    à celle d’une playlist musicale et qui correspondrait à des revues de presse. Il faut ainsi définir un certain nombre de critères de rapprochement
    et de tri des articles, mais aussi un outil performant permettant de faire cette sélection de façon rapide.

=======
    \textbf{Recherche dans la page :} la visionneuse aura un outil de recherche plein texte. L'utilisateur doit pouvoir effectuer une recherche à partir d’un mot présent dans un article ou un ensemble de mots de même racine ou phonétiquement similaires.
    La visionneuse fournira un ensemble d’articles et d'extraits répondant à la requête et mettra en avant les mots correspondants à la recherche pour facilement les repérer.
    L’outil de recherche sera adapté à un document structuré tel que de le XML (le contenu d’une page étant stocké dans ce format) et fournira
    un résultat quasi-instantané (inférieur à trois secondes).


    \textbf{Découper et identifier une page de journal :} la visionneuse structurera, grâce à un système de calques transparents, le contenu d’une page selon
    les articles, images ou tout autre élément intéressant. La détection du contenu est réalisée par un outil de l’équipe Intuidoc qui fournit un
    fichier XML contenant le découpage de la page du journal. La visionneuse permettra la définition de zones résistantes au zoom ou autres
    rotations et les calques transparents doivent pouvoir prendre la forme de polygones non convexes. En effet, les articles peuvent avoir des formes complexes.


    \textbf{Proposition d’articles proches :} cette fonctionnalité sera dans l’outil de visionnage. Sur la page de visualisation de document, l'utilisateur verra des suggestions d'articles en rapport à celui qu'il est en train de visualiser ; par exemple des articles sur le même sujet ou les éditions suivantes ou précédentes
    du même journal. De plus, L'utilisateur se verra proposer la création et la visualisation de revues de presse dont on peut rapporter la structure
    à celle d’une playlist musicale. Il faut ainsi définir un certain nombre de critères de rapprochement et de tri des articles, mais aussi un outil performant permettant de faire cette sélection de façon rapide.
>>>>>>> 1332c7553389d869a280204ebb3a682f6c01895e

    \textbf{Revues de presse et système collaboratif :} cette fonctionnalité est celle qui fait la différence entre notre site et les autres sites de consultation de documents que l'on peut trouver sur internet. Une revue de presse est un ensemble d'articles qui ont un thème en commun. Ces revues sont visibles pour tous les utilisateurs. Ces derniers pourront d'ailleurs ajouter eux-mêmes des articles à la revue. Idéalement, un tel espace d’entraide permettra aux utilisateurs de partager et de mettre en avant des articles.
		
		\textbf{Système de gestion d'utilisateur :} afin de développer ce système de revues de presse, l’application bénéficiera d’un système de gestion d’utilisateur. Les utilisateurs seront liés aux revues qu’ils ont créées. Le créateur aura la possibilité de modifier l'ordre des articles ainsi que d'en retirer de la revue et de supprimer la revue. Avoir un compte permettra aussi d'ajouter des articles à une liste de favoris ainsi que d'ajouter et de supprimer des tags d'articles.

    Dans un premier temps, nous allons décrire les fonctionnalités qui seront accessibles à l'utilisateur lors de l'arrivée sur la plateforme. La recherche dans la base puis la consultation de documents seront détaillées. Nous développerons par la suite le fonctionnement des revues de presse et de la gestion des profils utilisateurs. Nous finirons le rapport en établissant une planification initiale des tâches de développement.