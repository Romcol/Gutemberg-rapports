\section{Introduction}
\label{sec:intro}

	Lors du précédent rapport, nous avions fait état des objectifs du projet et de l'existant autour de celui-ci, et nous avions fait une première liste d'un certain nombre de technologies et d'outils qui nous permettraient de développer la plateforme. Nous avions aussi abordés succinctement les principes et les fonctionnalités que devra avoir l'application. 

	Dans ce rapport, le but est d'établir une liste quasi-exhaustive des spécifications fonctionnelles que la plateforme devra contenir, en présentant en détails celles-ci. On présentera donc chaque fonctionnalité en décrivant un scénario d'usage, l'intérêt de celle-ci dans l'application, expliquer pourquoi elle est nécessaires et en détailler le fonctionnement. On évaluera aussi l'importance de chaque fonctionnalité vis-à-vis du développement de l'application finale. Ce rapport sera aussi l'occasion d'établir une première planification des différentes versions de l'application, en fonction de l'importance et de la priorité affectée à chaque tâche, et de la durée prévue pour chacune des tâches. Nous préciserons comment sera organisé le développement de l'application, de manière à commencer à proposer une répartition des tâches suivant les ressources qui seront disponibles à l'étape du développement.

	Dans un premier temps, nous effectuerons donc un rappel du projet, des objectifs et du contexte de celui-ci. Nous allons par la suite détailler toutes les fonctionnalités de l'application, puis finalement proposer une organisation et une planification du projet.