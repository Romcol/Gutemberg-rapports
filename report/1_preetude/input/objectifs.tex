\section{Les Objectifs}
    \label{sec:objectifs}

    Nous allons découper les objectifs de l’application en plusieurs phases ; au fur et à mesure
    du développement les fonctionnalités vont évoluées, l’idée étant d’avoir à n’importe quel moment un produit fonctionnel.

    \subparagraph{Lire un journal :}
    fonctionnalité essentielle, un utilisateur doit pouvoir consulter un journal, en parcourir
    toutes ses pages. Les journaux peuvent avoir une taille très supérieur aux images standard,
    il faudra donc disposer des journaux sous plusieurs résolution, une faible résolution pour visionner gloablement
    le journal et une version haute résolution pour la lecture des articles.
    La navigation entre les pages du journal, et entre les journaus, nécessitera une visionneuse d’image performante.

    \subparagraph{Recherche avancée :}
    second point très important, l’utilisateur doit disposer d’une fonctionnalité de recherche
    qui va permettre de chercher des articles suivant la date, le sujet... Il faudra aussi que l’utilisateur
    puisse rechercher un article en particulier. Il nous faudra aussi une base de données performante
    pour que la recherche d’articles soit efficace. Cette base de données ne contiendra, au début,
    que quelques milliers d’éléments mais à pour vocation de contenir un ensemble important de journaux,
    quelques centaines de milliers voir quelques millions.

    \subparagraph{Recherche plein texte :}
    par la suite le visionnage de document devra être amélioré pour permettre une recherche plein texte.
    L’idée principale est de pouvoir, sur une page quelconque, pouvoir rechercher du texte. Si une (ou plusieurs)
    occurrence est trouvée, alors celle-ci sera mise en évidence sur l’image pour situer à l’utilisateur
    la présence de ce texte. Il sera d’un côté nécessaire d’avoir une visionneuse d’image qui permette facilement
    le ‘dessin’ d’éléments par-dessus l’image d’une part, et une recherche dans des documents structurés
    (le contenu d’une page étant stockée dans un xml) assez efficace pour obtenir un résultat quasiment instantané.

    \subparagraph{Découper et identifier une page de journal :}
    Il nous faudra aussi proposer un découpage à chaque page de journal selon les articles, images ou
    tout autre élément structurel interessant, dans le but de permettre à l’utilisateur d’afficher directement
    un élément précis. Cette fonctionnalité nécessite les mêmes prérequis que la tâche précédente,
    c’est à dire une visionneuse performante et configurable qui permette le dessin sur les images dans
    le but d’afficher des informations à l’utilisateur. La détection du contenu d’une page de journal est
    laissée à un outil de l’équipe Intuidoc qui nous fournit un fichier XML contenant la structure de
    la page de journal ainsi que le contenu des articles.

    \subparagraph{Proposition articles proches :}
    Ensuite, il sera intéressant d’ajouter des propositions de lecture standard, redirigeant vers certains
    journaux en lien avec celui que l’utilisateur est en train de lire (édition précédente/suivante,
    autre journal pour la même date ...), pour proposer une navigation plus fluide entre les journaux pour un utilisateur.

    \subparagraph{Navigation entre journaux avec les parcours thématiques :}
    enfin, la dernière fonctionnalité assez conséquente est de permettre aux utilisateurs
    de créer des parcours thématiques, regroupant des journaux ou des articles autour
    d’un ou plusieurs éléments spécifiques choisis par l’utilisateur. Cela permettra
    par la suite de proposer une navigation encore plus fluide entre les divers journaux,
    en proposant aux personnes lisant un certain journal, d’autres journaux faisant parti
    d’un même parcours thématique. Il est important dans ce contexte de permettre à plusieurs utilisateurs
    de travailler sur un même parcours pour modifier, ajouter ou enlever des éléments à celui-ci.
    Cet espace d’entraide et de création pourra se faire par le biais d’un wiki, qui est un environnement
    plutôt adéquat pour ce genre de fonctionnalités.
