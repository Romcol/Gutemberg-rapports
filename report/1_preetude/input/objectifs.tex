\section{Les Objectifs}
	\label{sec:objectifs}

    Le projet se résume donc à la réalisation d’une plateforme qui permettra aux utilisateurs
    de consulter des informations mais aussi d’interagir avec les éléments de l’application.
    En effet, si la tâche principale est de pouvoir accéder aux divers documents numérisés afin
    de consulter ceux-ci, l’utilisateur doit pouvoir accéder à d’autres fonctionnalités.
    Ainsi il sera possible d'apporter des modifications sur certaines annotations
    extraites des images numérisées sur divers documents pour apporter des informations
    supplémentaires sur ceux-ci ou en corriger certaines. Mais il sera aussi nécessaire
    de permettre la création et la gestion de parcours thématiques par les utilisateurs
    afin d’offrir une navigation collaborative dans la presse ancienne.

    On peut donc résumer les objectifs de l’application de façon très concise :

    \begin{itemize}
        \item{Accéder aux documents numérisés}
        \item{Recherche à travers les documents}
        \item{Geérer des annotations pour chaque document}
        \item{Créer et gérer des parcours thématiques}
    \end{itemize}

    L’objectif principal est de gérer les données de milliers puis de millions de documents
    qui sont – et seront – numérisés et mis à disposition à travers la plateforme.
    Si au départ celle-ci ne contiendra que quelques milliers de documents, la base de données
    qui sera utilisée devra toujours être accessible rapidement afin de limiter les temps d’attente;
    au final un des objectifs importants sera d’optimiser la base de donnée reliée à l’application.

    Bien évidemment, la recherche d’éléments constitue la fonctionnalité clé de l’application,
    puisqu’il est primordial de pouvoir chercher et/ou retrouver efficacement un document
    en possédant certaines informations ou en ayant un domaine de recherche. Il sera donc nécessaire
    de proposer des champs de recherche pouvant couvrir tous les besoins, mais aussi que celle-ci
    se fasse assez rapidement. La recherche plein texte dans un document ou sur une page
    sera aussi important afin d’avoir un visuel direct sur des éléments que l’on pourrait rechercher
    dans une page, d’autant plus que certains journaux ont une résolution de 5000x4000 pixels et
    qu’il serait trop embêtant pour l’utilisateur de devoir chercher l’information de lui-même dans la page.

    Le but de pouvoir gérer les annotations, qui seront en partie initialisées après la récupération
    des informations du document, sera de permettre aux utilisateurs de corriger les données quand
    le résultat de l’analyse ne semble pas très pertinent, ou de rajouter des commentaires ou
    de nouvelles informations afin de fournir le maximum d’informations possibles et utiles aux utilisateurs.

    Enfin, le dernier objectif mais pas des moindres, sera donc la gestion de parcours thématiques,
    créés par et pour les utilisateurs. Le but est de rassembler autour d’un même évènement,
    d’un même journal ou d’une même personnalité, divers articles sélectionnés par ces mêmes utilisateurs
    afin de proposer, entre autres, des pistes de lecture. Le défi ici est de proposer de manière simple,
    automatique et pratique la création, la modification et le partage de ces parcours afin que cela
    ne devienne pas une besogne pour ceux qui souhaitent partager de l’information.
