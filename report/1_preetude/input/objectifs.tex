\section{Les Objectifs}
	\label{sec:objectifs}

    On va découper les objectifs de l’application en plusieurs phases ; au fur et à mesure
    du développement les fonctionnalités vont évoluées, l’idée étant d’avoir à n’importe quel moment un produit fonctionnel.

    \subparagraph{Lire un journal :} 
    fonctionnalité essentielle, un utilisateur doit pouvoir consulter un journal, en parcourir
    toutes ses pages. Les journaux peuvent avoir une taille très supérieur aux images standard,
    il faudra donc pouvoir avoir des « miniatures » des images qui devront être zoomable pour
    permettre à un utilisateur d’arriver à une taille lisible tout en permettant d’avoir avant
    cela une vue d’ensemble de la page. Une visionneuse d’images permettant de naviguer entre
    les images et permettant cette action devra être nécessaire.
    
    \subparagraph{Recherche avancée :}
    second point très important, l’utilisateur doit disposer d’une fonctionnalité de recherche
    qui va permettre de chercher des journaux suivant le journal, la date, le sujet... Pouvoir rechercher
    directement un article sera aussi quelque chose d’intéressant pour l’utilisateur. Enfin il faudra pouvoir
    proposer un tri sélectif, par journal et par date par exemple. Si au départ très peu d’éléments seront
    présents dans la base de données, on imagine par la suite avoir de très nombreux documents archivés
    (plus d’un million), avec des enregistrements très gros, il sera nécessaire d’avoir une base de données
    performante pour que l’affichage des résultats pour l’utilisateur soit rapide.

    \subparagraph{Recherche plein texte :}
    par la suite le visionnage de document devra être amélioré pour permettre une recherche plein texte.
    L’idée principale est de pouvoir, sur une page quelconque, pouvoir rechercher du texte. Si une (ou plusieurs)
    occurrence est trouvée, alors celle-ci sera mise en évidence sur l’image pour situer à l’utilisateur
    la présence de ce texte. Il sera d’un côté nécessaire d’avoir une visionneuse d’image qui permette facilement
    le ‘dessin’ d’éléments par-dessus l’image d’une part, et une recherche dans des documents XML
    (le contenu d’une page étant stockée dans un xml) assez efficace pour obtenir un résultat quasiment instantané.

    \subparagraph{Découper et identifier une page de journal :}
    ajout supplémentaire lors de la lecture de journaux, proposer un découpage de chaque page lues avec les articles,
    images ou tout élément intéressant à l’utilisateur pour permettre à celui-ci de directement zoomer sur
    un article précis, ou de montrer simplement le découpage de la page afin de voir son organisation et avoir
    une liste des éléments existants. Cela nécessitera comme pour la tâche précédente une visionneuse
    qui permette de ‘dessiner’ sur l’image pour présenter les éléments à l’utilisateur, et d’un analyseur
    de document XML assez vaste qui permette de reconnaître l’architecture de n’importe quelle page,
    et qui ne rajoute pas trop de temps de traitement pour l’affichage de chaque page

    \subparagraph{Proposition articles proches :}
    ensuite, il sera intéressant d’ajouter des propositions de lecture standard, redirigeant vers certains
    journaux en lien avec celui que l’utilisateur est en train de lire (édition précédente/suivante,
    autre journal pour la même date ...), pour proposer une navigation plus fluide entre les journaux pour un utilisateur.

    \subparagraph{Navigation entre journaux avec les parcours thématiques :}
    enfin dernière fonctionnalité assez conséquente, permettre aux utilisateurs de créer des parcours thématiques,
    regroupant des journaux ou des articles autour d’un ou plusieurs éléments spécifiques choisis par l’utilisateur.
    Cela permettra par la suite de proposer une navigation encore plus fluide entre les divers journaux,
    en proposant aux personnes lisant un certain journal d’autres journaux faisant parti d’un même parcours thématique.
    Il est important dans ce contexte de permettre à plusieurs utilisateurs de travailler sur le même parcours
    pour modifier, ajouter ou enlever des éléments à celui-ci. Cette espace d’entraide et de création pourra
    se faire par le biais d’un wiki, qui est un environnement plutôt adéquat pour ce genre de fonctionnalités.
