\section{Les Objectifs}
	\label{sec:objectifs}

    Le projet se résume donc à la réalisation d’une plateforme qui permettra aux utilisateurs de consulter des informations mais aussi d’interagir avec
    les éléments de l’application. En effet, si la tâche principale est de pouvoir accéder aux divers documents numérisés afin de consulter ceux-ci,
    l’utilisateur doit pouvoir accéder à d’autres fonctionnalités. Ainsi il sera possible de modifier les premières annotations extraites des images numérisées
    sur divers documents pour apporter des informations supplémentaires sur ceux-ci ou en corriger certaines, mais il sera aussi nécessaire de permettre
    la création et la gestion de parcours thématiques par les utilisateurs afin d’offrir une navigation collaborative dans la presse ancienne.

    On peut donc résumer les objectifs de l’application de façon très concise :
    \begin{itemize}
        \item{Accéder aux documents numérisés}
        \item{Recherche à travers les documents}
        \item{Gestion des annotations de chaque document}
        \item{Création et gestion de parcours thématiques}
    \end{itemize}

    L’objectif principal est de gérer les données des millions de documents qui sont – et seront – numérisés à travers la plateforme. Si au départ celle-ci
    ne contiendra que quelques milliers de documents, la base de données qui sera utilisée devra toujours être accessible rapidement afin de limiter les temps d’attente ;
    au final une des objectifs importants sera d’optimiser la base de donnée reliée à l’application.

    Le but de pouvoir gérer les annotations, qui seront en partie initialisées après la récupération des informations du document, sera de permettre aux utilisateurs
    de corriger les données, ou de rajouter des commentaires ou de nouvelles informations afin, dans un esprit de collaboration, de fournir le maximum d’informations
    possibles et utiles aux utilisateurs.

    Enfin, dernier objectif mais pas des moindre, sera donc la gestion de parcours thématiques, créés par et pour les utilisateurs, afin de rassembler autour d’un même
    évènement, d’un même journal ou d’une même personnalité divers articles sélectionnés par ces mêmes utilisateurs afin de proposer, entre autres, des pistes de lecture.
    Le défi ici est de proposer de manière simple, automatique et pratique la création, la modification et le partage de ces parcours afin que cela ne devienne pas une
    besogne ceux qui souhaitent partager de l’information.
