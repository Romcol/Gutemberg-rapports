\section{Solutions techniques}
\label{sec:technique}

    \subsection{Type de la plateforme}
    \label{subsec:plateforme}
    Il n’y a aucune contrainte quant à la réalisation de la plateforme, si ce n’est qu’elle doit être accessible pour tous types
    d’utilisateurs : PC (Windows/Mac/Linux), Tablettes (Android/iOS) voir même Smartphones (Android/iOS/Windows phone/Blackberry…).
    Il serait envisageable d’imaginer réaliser une application multiplateforme qui puisse être installée et fonctionne sur tous
    types de plateforme, cependant cela imposera de très nombreuses contraintes techniques, il faudra sans cesse penser à cette
    comptabilité et les tests n’en seraient que plus embêtant. Il est bien évidemment exclus de réaliser une application unique
    à chaque plateforme, ne serait-ce que pour le temps de développement mais aussi lors de la reprise de celle-ci dans le futur
    qui rendrait ceci très difficile.

    Il semble donc naturel de partir sur le développement d’une plateforme web, ce qui permettra de la rentre compatible avec
    n’importe quel navigateur, qui facilitera la communication client/serveur (qui aurait pu poser des problèmes si l’application
    installée sur la machine de l’utilisateur devait communiquer avec un serveur). De plus la gestion de comptabilité entre les
    différents formats de visualisation sera réalisée grâce aux outils de design que permet une application web, cela facilitera
    l’implémentation.

    \subsection{Technologie de la platefome}
    \label{subsec:technologie}
    Il existe plusieurs langages qui permettraient de développer une application web comme
    celle-ci. Chaque langage a ses propres avantages et inconvénients, nous allons ici présenter de façon
    exhaustive chaque langage possible avant de lister les avantages et inconvénients de chacun. Ces
    choix feront lui d’un débat lors d’une réunion du groupe afin de décider d’une technologie à utiliser
    pour commencer à la prendre en main et réaliser une ébauche afin de voir ce que l’on sera capable
    de faire de nous-même avec avant le développement de la plateforme.

        \subsubsection{PHP}
        \label{subsubsec:php}
        PHP est utilisé sur plus de 75\% des serveurs web ; il y a plus de 27 millions de sites web qui
        sont codés en PHP, ce qui correspond à 50\% des sites web dans le monde. C’est un langage
        polyvalent et il est considéré comme le langage web standard. Ce langage est notamment privilégié
        car il est simple à apprendre et il est très simple d’installer un serveur web fonctionnant sous PHP,
        que ce soit sous Windows, Max ou Linux. Il est aussi à l’origine de quelques plateformes comme
        WordPress, Wikipedia, et est même utilisé dans une partie de Facebook. PHP a été imaginé pour
        créer des sites webs. 

        De plus de très nombreux frameworks imaginés pour le développement de sites web existent
        et fournissent des outils spécifiques permettant d’améliorer l’efficacité du code. La bibliothèque de
        PHP est aussi bien riche et diversifiée ; ce qui nous permet de résoudre presque tous les problèmes
        qui peuvent arriver en réalisant un site web.

        \subsubsection{Node.js / Javascript}
        \label{subsubsec:node}
        A ne pas confondre avec Java, Javascript est un “maître” qui nous permet de faire des
        interactions dans notre site web. C’est un langage particulier qui fonctionne en fonction de la
        navigation de l’utilisateur. Il nous permet d’interagir avec le site web avec des évènements et permet
        de mettre à jour dynamiquement des éléments sans avoir à recharger une page. Pour ce projet,
        l’utilisateur interagira avec divers éléments très souvent et il serait très embêtant pour lui de devoir
        attendre le rechargement de pages. Le langage propose notamment un framework très connu,
        JQuery, qui permet d’écrire beaucoup moins de code tout en améliorant l’interactivité de la
        plateforme.

        Node.js est une technologie très récente, encore jeune, permet tout simplement de gérer un
        serveur entièrement en Javascript. En effet il était de base nécessaire de manipuler PHP côté serveur
        et Javascript côté client. On a ici l’avantage de gérer ces deux parties dans un seul langage, ce qui
        facilite énormément la reprise d’un projet. De plus, Node.js montre aussi de très bonnes
        performances qui restent stables quel que soit le nombre d’utilisations, il faut cependant avec une
        très bonne connaissance de Javascript pour gérer le serveur par cette technologie.

        \subsubsection{Ruby}
        \label{subsubsec:ruby}
        Ruby est accompagné d’un framework, Ruby on Rails, pour developper un site. Il est
        fortement utilisé dans quelques grands sites web comme Groupon, Shopify, … et est utilisé pour faire
        le front-end du site web social Twitter. Ruby est open-source, c’est un langage orienté objet et est
        interprété par le serveur avant d’envoyer du code HTML au navigateur de l’utilisateur (comme PHP).

        Pourtant, Ruby a ses points forts: développement des applications très rapides, moins de répétition
        de code et le temps de traitement est plus court que PHP. Ruby on Rails a été imaginé pour faire des
        sites web compliqués, et comme Perl, plus performant. Néanmoins, Ruby a aussi un point faible ;
        contrairement à PHP il existe un moins grand nombre de serveurs qui supportent Ruby par défaut.

        \subsubsection{.NET}
        \label{subsubsec:dotnet}
        .NET ou ASP.net - un produit de Microsoft pour participer a la compétition de la programmation site web.
        Cette platforme est bien utilisée dans les entreprises mais individuellement, on ne l’utilise pas pour
        réaliser un site web. Car on doit avoir un host Windows particulier pour exécuter une application codée
        en ASP.net, ça revient a peu près le même problème avec platforme Ruby. De plus, apprentissage de ASP.net
        est difficile, on doit faire beaucoup plus d’efforts que l’autre plateforme pour obtenir un produit qui vaut la même valeur.


        \subsubsection{Récapitulatif}
        \label{recap}
        \begin{tabular}{|l|l|}
            \hline
            Solution & Qualités/Défauts \\ \hline
            \multirow{8}{*}{PHP} & + Fonctionne sur tous les serveurs acceptant PHP (presque tous) \\
                & + Pas besoin de différencier les navigateurs du marché \\
                & + Langage web le plus répandu et le plus utilisé \\
                & + De nombreux frameworks disponibles \\
                & - Nombreuses failles de sécurité \\
                & - Langage interprété problème de performances si beaucoup de clients \\
                & - Non typé \\
                & - Débogage difficile \\ \hline
            \multirow{6}{*}{Python} & + Syntaxe simple et facilement lisible \\
                & + Nombreuses bibliothèques disponibles \\
                & + Communauté énorme et active \\
                & + Permet d’intégrer un code écrit dans un autre langage \\
                & - Détection d’erreur à l’exécution \\
                & - Langage interprété : problème de performances si beaucoup de clients \\ \hline
            \multirow{6}{*}{Ruby} & + Langage concis, code plus lisible et reprise plus aisée \\
                & + Langage fortement orienté objet \\
                & + Gestion de librairie avec Bundler \\
                & + Communauté très active \\
                & + Framework Ruby on Rails \\
                & - Langage interpreté: problème de performances si beaucoup de clients \\ \hline
            \multirow{5}{*}{Node.js} & + Très performant, mëme si de nombreuses utilisations \\
                & + Temps de traitement et de changement faibles \\
                & + Communauté très active \\
                & - Nécessite une bonne maîtrise du langage Javascript \\
                & - Technologie jeune, pas de preuves de durée sur le long terme \\ \hline
            \multirow{6}{*}{.Net} & + Langage compilé, éxecution plus rapide \\
                & + Possibilité d'utiliser plusieurs langages \\
                & + Garbage collector \\
                & - Ne fonctionne que sur Microsoft Server \\
                & - IDE très lourd \\ 
            \hline
        \end{tabular}

    \subsection{Gestion de la base de données}
    \label{subsec:bdd}

    Notre plateforme disposera aussi d’une base de données afin de stocker les données des documents numérisés
    ainsi que celle des comptes utilisateurs. On distingue d’abord deux types de système de gestion de base de données.
    D’un côté, les bases de données relationnelles. Ce sont des bases où les données sont organisées sous forme de tables
    (ou relation) à deux dimensions. C’est le système de base de données le plus utilisé. Les bases de données sont manipulées
    à l’aide du langage SQL. La présence d’un schéma relationnel qui structure les données permet de faire
    des requêtes riches, ce qui permet de manipuler efficacement les données.

    D’un autre côté, le NoSQL qui comprend tous les systèmes de base de données qui ne se base pas sur une
    architecture relationnelle. Le but est de gagner en simplicité notamment en s’affranchissant du schéma relationnel
    qui est fixé lors de la création de la table et qui contraint l’ajout de nouvelles données. Un exemple de modèle en NoSQL,
    c’est un système de clé-valeur. La base de données peut se résumer alors à un tableau associatif avec des millions d’entrées.

    Cependant, cette simplicité réduit la richesse des requêtes et déplace la complexité intrinsèque de la requête vers
    la logique de l’application. En revanche, ce système de base de données gère mieux le cas où les données sont réparties sur plusieurs serveurs.

        \subsubsection{Liste des solutions}
        \label{subsubsec:bddsolutions}

        \begin{center}
        \begin{tabular}{|l|l|l|}
            \hline Type & Solutions & Qualités/Défauts \\ \hline
            \multirow{20}{*}{SQL} & \multirow{4}{*}{Oracle} & + Très performant \\
            & & + Très bon temps de réponse \\
            & & + Fournis beaucoup d'outils \\
            & & - Licence propriétaire \\ \cline{2-3}
            & \multirow{4}{*}{MySQL} & + Gratuit \\
            & & + Très performant sur des petits volumes de données \\
            & & + Très performant avec un faible accès \\
            & & - Performance très faible sur des grosses bases de données \\ \cline{2-3}
            & \multirow{6}{*}{PostgreSQL} & + Communauté très active \\
            & & + Respecte la norme SQL2003 \\
            & & + Beaucoup d’outils disponibles \\
            & & + Gère très bien de gros volumes de données grâce à son optimiseur \\
            & & - Moins performant que d’autres sur des petits volumes de données \\
            & & - Nécessite une très bonne configuration et une base entretenue \\ \cline{2-3}
            & \multirow{6}{*}{Windows Access} & + Facile d’utilisation et de maintenance \\
            & & + Très rapide à mettre en œuvre \\
            & & - Licence propriétaire \\
            & & - N’est pas fait pour traiter beaucoup de données \\
            & & - Données stockées dans un fichier \\
            & & - Utilisation restreinte à Windows \\ \hline
            \multirow{8}{*}{NoSQL} & \multirow{4}{*}{MongoDB} & + Solution NoSQl la plus populaire \\
            & & + Permet de manipuler des objets structurés \\
            & & + Ne nécessite pas de schéma prédéfini des données \\
            & & - Ne respecte pas l’ensemble des propriétés A.C.I.D. \\ \cline{2-3}
            & \multirow{4}{*}{Cassandra} & + Très performants pour des bases distribuées \\
            & & + Propriétés ACID respectée par ligne \\
            & & - Schéma défini à l’avance car système orienté colonne \\
            & & - Rigide lors de la recherche \\
            \hline
        \end{tabular}
        \end{center}

    \subsection{Besoins de la plateforme}
    \label{subsec:besoinplateforme}
    Les caractéristiques mises en avant précédemment sont d’ordre général. Nous allons à présent regarder
    vis-à-vis des particularités de notre base de données.
    Les données que nous manipulerons seront nombreuses, l’application est destinée entre autres
    à des services d’archives départementales qui ont des dizaines de milliers de documents numérises
    et ce nombre s’accroit. Ce sont aussi des données plutôt volumineuses (plusieurs Mo le document).
    Cela signifie que nous avons besoin d’une solution qui soit optimisée pour ce genre de cas tout en
    restant performant car l’utilisateur ne doit pas attendre longtemps l’affichage du document.
    Donc, MySQL ne figure pas dans les meilleures solutions pour cette particularité.

    La solution devra aussi nous permettre d’implémenter le concept de document numérisé de manière
    à ne pas rendre le travail de manipulation des données trop complexe. De même, nous aurons besoin
    de faire des recherches en fonction du titre, de l’année… Le langage devra permettre cela.
    Cependant, nous pouvons accepter les langages qui ne proposent qu’une recherche rudimentaire
    car nous pourrons implémenter une recherche plus sophistiquée avec le langage de l’application.
    C’est le cas, par exemple, de MongoDB.

    Il faut prendre aussi une particularité intrinsèque au projet : la connaissance du langage.
    En effet, utiliser un langage que nous maitrisons déjà nous permettrait de gagner du temps.
    C’est le cas du SQL que nous avons vu lors de notre cursus scolaire. Cependant apprendre un
    nouveau langage est toujours un apport ludique. Les projets sont l’occasion d’utiliser des
    langages différents de ceux enseignés dans le cursus. Mais il ne faut pas non plus choisir
    un langage trop obscur car le projet sera probablement poursuivi les années suivantes.
    Il serait préférable que le langage choisi soit motivant pour les prochains groupes qui travailleront sur le projet.
    Enfin, certaines solutions sont des licences propriétaires. Cela signifie que l’utiliser
    nécessite d’acheter une licence. Nous aimerions éviter d’avoir à utiliser ce genre de solutions
    non seulement pour une raison financière mais aussi parce que nous préférerions  que notre plateforme soit open source.


