\section{Solutions techniques}
\label{sec:technique}

    \subsection{Type de la plateforme}
    \label{subsec:plateforme}
    Il n’y a aucune contrainte quant à la réalisation de la plate-forme, si ce n’est qu’elle doit être
    accessible pour tous types d’utilisateurs : PC (Windows/Mac/Linux), Tablettes (Android/iOS) voir
    même Smartphones (Android/iOS/Windows phone/Blackberry…). Il serait envisageable d’imaginer réaliser
    une application multi-plate-forme qui puisse être installée et fonctionne sur tous types de plates-formes,
    cependant cela imposerait de très nombreuses contraintes techniques, il faudrait sans cesse penser
    à cette comptabilité et les tests n’en seraient que plus embêtant. Il est bien évidemment exclu de réaliser
    une application unique à chaque plate-forme, car le temps de développement disponible n’est pas suffisant.

    Il semble donc naturel de partir sur le développement d’une plate-forme web, ce qui permettra de la rendre
    compatible avec n’importe quel navigateur et qui facilitera la communication client/serveur. De plus la gestion
    de compatibilité entre les différents formats de visualisation sera réalisée grâce aux outils de design
    que permet une application web.

    \subsection{Technologie de la platefome}
    \label{subsec:technologie}
    Il existe plusieurs langages qui permettraient de développer une application web comme celle-ci.
    Chaque langage a ses propres avantages et inconvénients, nous allons ici présenter de façon non-exhaustive
    certaines langages possibles avant de lister les avantages et inconvénients de chacuns. Ces choix feront
    lieu de tests (proof of concept) afin de décider d’une technologie à utiliser pour commencer à réaliser
    une ébauche afin de voir ce que l’on sera capable de faire de nous-mêmes avec avant le développement de la plate-forme.

        \subsubsection{PHP}
        \label{subsubsec:php}
        PHP est utilisé sur plus de 75\% des serveurs web ; il y a plus de 27 millions de sites web qui sont codés en PHP,
        ce qui correspond à 50\% des sites web dans le monde. C’est un langage polyvalent et il est considéré comme
        le langage web standard. Ce langage est notamment privilégié car il est simple à apprendre et il est très
        simple d’installer un serveur web fonctionnant sous PHP, que ce soit sous Windows, Mac ou Linux. Il est
        aussi à l’origine de quelques plates-formes comme WordPress ou Wikipedia, et est même utilisé dans une
        partie de Facebook. PHP a été imaginé pour créer des sites webs.

        De plus de très nombreux frameworks imaginés pour le développement de sites web existent et fournissent
        des outils spécifiques permettant d’améliorer l’efficacité du code. La bibliothèque de PHP est aussi bien
        riche et diversifiée ; ce qui nous permet de résoudre presque tous les problèmes qui peuvent arriver
        en réalisant un site web.

        \pagebreak

        \paragraph{Framework PHP}
        \label{par:framworkPHP}

            \subparagraph{Zend Framework :}
            \label{subpar:zend}
            Zend Framework est créé et maintenu par l'entreprise qui a créé le langage PHP, on peut imaginer
            que ce framework continu à exister et à évoluer tant que PHP fera de même. Au départ, il n'y avait
            pas assez de bibliothèques pour satisfaire le besoin des développeurs et il possédait tellement
            de défauts que ceux-ci devaient utiliser d’autres framework en même temps. Au fur et à mesure,
            de nouvelles bibliothèques ont été ajoutées et c'est la raison pour laquelle des grandes entreprises comme IGN.com,
            AskMen.com et IBM ont choisi Zed pour développer des applications en ligne et aujourd’hui  sa bibliothèque
            est très riche. Par ailleurs, ce framework n’a pas de gros problèmes de sécurité, et permet d’éviter
            des erreurs facilement faîtes en PHP. Il est aussi aisé de développer en utilisant l’architecture
            Modèle - Vue - Contrôleur, qui est une bonne architecture pour séparer les tâches et
            le développement / reprise de l’application.

            \subparagraph{Yii framework}
            \label{subpar:yii}
            Yii est un framework PHP très performant pour développer des applications web. Il utilise le principe
            Modèle - Vue - Contrôleur qui est plutôt utilisé dans la programmation site web et qui permet aux développeurs
            d’effectuer facilement des modifications sans modifier du code annexe. Il permet non seulement de
            développer des applications plus rapidement en générant les opérations basiques, mais permet aussi de
            maintenir facilement l’application. “Migration Utility” est intégrée, et permet de changer facilement
            de base de données, et facilite la mise a jour des différentes versions de l’application. Yii fournit
            aussi un bon invité de commande qui permet de créer des CRON assez compliqués en utilisant une application
            logique intégrée qui évite de les réécrire.

            \subparagraph{Symfony}
            \label{subpar:symfony}
            Symfony est un des framework de PHP les plus populaires pour créer des sites web et des applications web :
            on recense plus de 1000 contributeurs dans le projet Symfony Framework et pas moins de 300 000 développeurs
            qui utilisent Symfony. Symfony est développé par l'entreprise SensioLabs qui est établie depuis plus que 12 ans,
            sa permanence est donc bien assurée. Récemment Symfony s’est vu offrir une “Open-Source MIT license”
            qui n'impose aucune contrainte et permet de développer ses propres applications. Ce framework utilise aussi
            la structure MVC, il possède tout ce dont on a besoin sur un framework : rapidité, flexibilité, des composants réutilisables, ...

        \subsubsection{Node.js / Javascript}
        \label{subsubsec:node}
         ne pas confondre avec Java, Javascript est un  langage qui nous permet de faire des interactions dans notre site web.
         C’est un langage particulier qui fonctionne en fonction de la navigation de l’utilisateur. Il nous permet d’interagir
         avec le site web avec des évènements et permet de mettre à jour dynamiquement des éléments sans avoir à recharger une page.
         Pour ce projet, l’utilisateur interagira avec divers éléments très souvent et il serait très embêtants pour lui
         de devoir attendre le rechargement de pages. Le langage propose notamment un framework très connu, JQuery,
         qui permet d’écrire beaucoup moins de code tout en améliorant l’interactivité de la plateforme.

         Node.js est une technologie très récente, encore jeune, et permet tout simplement de gérer un serveur entièrement en Javascript.
         En effet il était de base nécessaire de manipuler PHP côté serveur et Javascript côté client. On a ici l’avantage
         de gérer ces deux parties dans un seul langage, ce qui facilite énormément la reprise d’un projet. De plus,
         Node.js montre aussi de très bonnes performances qui restent stables quel que soit le nombre d’utilisations,
         il faut cependant avoir une très bonne connaissance de Javascript pour gérer le serveur avec cette technologie.

        \subsubsection{Ruby}
        \label{subsubsec:ruby}
        Ruby est accompagné d’un framework, Ruby on Rails, pour développer les sites web. Il est fortement utilisé
        dans quelques grands sites web comme Groupon ou Shopify, et est utilisé pour faire le front-end du site web
        social Twitter. Ruby est open-source, c’est un langage orienté objet et est interprété par le serveur avant
        d’envoyer du code HTML au navigateur de l’utilisateur (comme PHP). Ruby a ses points forts: développement
        des applications très rapides, moins de répétition de code et le temps de traitement est plus court que PHP.
        De plus, Ruby on Rails a été imaginé pour faire des sites web compliqués et performants.

        \subsubsection{.NET}
        \label{subsubsec:dotnet}
        .NET ou ASP.net est un produit de Microsoft créé en 2002 qui visait à concurrencer les autres technologies web.
        C’est une technologie qui est utilisée dans de nombreuses entreprises, et on recense maintenant un grand nombre
        d’experts .NET. Cependant le principal défaut de ce langage est qu’il est propriétaire; en effet il n’est compatible
        qu’avec Microsoft Server, et donc uniquement avec Windows.

        \subsubsection{GO}
        \label{subsubsec:go}
        Go est un langage créé par Google et apparu en 2009. C’est un langage compilé qui a la particularité d’être
        rapide à l’exécution et de consommer très peu de mémoire. Il est donc adapté pour des sites à haute fréquentation.
        Il propose également des fonctionnalités de programmation concurrente plus simple à mettre en place que celles
        de ses concurrents.  Golang est très adapté au développement de services web sans avoir besoin de framework,
        et ce grâce aux nombreux packages natif mis à disposition du développeur. Pour finir, Go est un langage facile
        à apprendre, agréable à utiliser et permet un développement rapide.

        \subsubsection{Récapitulatif}
        \label{recap}
        \begin{tabular}{|l|l|}
            \hline
            Solution & Qualités/Défauts \\ \hline
            \multirow{8}{*}{PHP} & + Fonctionne sur tous les serveurs acceptant PHP (presque tous) \\
                & + Pas besoin de différencier les navigateurs du marché \\
                & + Langage web le plus répandu et le plus utilisé \\
                & + De nombreux frameworks disponibles \\
                & - Nombreuses failles de sécurité \\
                & - Langage interprété problème de performances si beaucoup de clients \\
                & - Non typé \\
                & - Débogage difficile \\ \hline
            \multirow{6}{*}{Ruby} & + Langage concis, code plus lisible et reprise plus aisée \\
                & + Langage fortement orienté objet \\
                & + Gestion de librairie avec Bundler \\
                & + Communauté très active \\
                & + Framework Ruby on Rails \\
                & - Langage interpreté: problème de performances si beaucoup de clients \\ \hline
            \multirow{5}{*}{Node.js} & + Très performant, mëme si de nombreuses utilisations \\
                & + Temps de traitement et de changement faibles \\
                & + Communauté très active \\
                & - Nécessite une bonne maîtrise du langage Javascript \\
                & - Technologie jeune, pas de preuves de durée sur le long terme \\ \hline
            \multirow{6}{*}{.Net} & + Langage compilé, éxecution plus rapide \\
                & + Possibilité d'utiliser plusieurs langages \\
                & + Garbage collector \\
                & - Ne fonctionne que sur Microsoft Server \\
                & - IDE très lourd \\ \hline
            \multirow{5}{*}{GO} & + Langage compilé, éxécution  très rapide \\
                & + Grabage collector, pas de gestion de mémoire \\
                & + Facile à apprendre \\
                & + Adapté au développement de services web \\
                & - Peu mature même si utilisé par des grands sites internets \\
            \hline
        \end{tabular}

    \subsection{Gestion de la base de données}
    \label{subsec:bdd}

   Notre plateforme disposera aussi d’une base de données afin de stocker les données des documents numérisés ainsi
   que celles des comptes utilisateurs. On distingue d’abord deux types de système de gestion de base de données.

   D’un côté, les bases de données relationnelles. Ce sont des bases où les données sont organisées sous forme
   de tables (ou relation) à deux dimensions. C’est le système de base de données le plus utilisé. Les bases
   de données sont manipulées à l’aide du langage SQL. La présence d’un schéma relationnel qui structure les données
   permet de faire des requêtes riches, et de manipuler efficacement les données.

   D’un autre côté, le NoSQLi, qui comprend tous les systèmes de base de données qui ne se basent pas sur
   une architecture relationnelle. Le but est de gagner en simplicité, notamment en s’affranchissant
   du schéma relationnel qui est fixé lors de la création de la table et qui contraint l’ajout de nouvelles données.
   Un exemple de modèle en NoSQL, c’est un système de clé-valeur. La base de données peut se résumer alors
   à un tableau associatif avec des millions d’entrées. Cependant, cette simplicité réduit la richesse des requêtes
   et déplace la complexité intrinsèque de la requête vers la logique de l’application. En revanche, ce système de base
   de données gère mieux le cas où les données sont réparties sur plusieurs serveurs.

        \subsubsection{Liste des solutions}
        \label{subsubsec:bddsolutions}

        \begin{center}
        \begin{tabular}{|l|l|l|}
            \hline Type & Solutions & Qualités/Défauts \\ \hline
            \multirow{20}{*}{SQL} & \multirow{4}{*}{Oracle} & + Très performant \\
            & & + Très bon temps de réponse \\
            & & + Fournis beaucoup d'outils \\
            & & - Licence propriétaire \\ \cline{2-3}
            & \multirow{4}{*}{MySQL} & + Gratuit \\
            & & + Très performant sur des petits volumes de données \\
            & & + Très performant avec un faible accès \\
            & & - Performance très faible sur des grosses bases de données \\ \cline{2-3}
            & \multirow{6}{*}{PostgreSQL} & + Communauté très active \\
            & & + Respecte la norme SQL2003 \\
            & & + Beaucoup d’outils disponibles \\
            & & + Gère très bien de gros volumes de données grâce à son optimiseur \\
            & & - Moins performant que d’autres sur des petits volumes de données \\
            & & - Nécessite une très bonne configuration et une base entretenue \\ \cline{2-3}
            & \multirow{6}{*}{Windows Access} & + Facile d’utilisation et de maintenance \\
            & & + Très rapide à mettre en œuvre \\
            & & - Licence propriétaire \\
            & & - N’est pas fait pour traiter beaucoup de données \\
            & & - Données stockées dans un fichier \\
            & & - Utilisation restreinte à Windows \\ \hline
            \multirow{15}{*}{NoSQL} & \multirow{5}{*}{MongoDB} & + Solution NoSQl la plus populaire \\
            & & + Permet de manipuler des objets structurés \\
            & & + Ne nécessite pas de schéma prédéfini des données \\
            & & - Ne respecte pas l’ensemble des propriétés A.C.I.D. \\
            & & - Difficilement scalable \\ \cline{2-3}
            & \multirow{4}{*}{Cassandra} & + Très performants pour des bases distribuées \\
            & & + Propriétés ACID respectée par ligne \\
            & & - Schéma défini à l’avance car système orienté colonne \\
            & & - Rigide lors de la recherche \\ \cline{2-3}
            & \multirow{6}{*}{Couchbase} & + Facilement scalable \\
            & & + Permet de manipuler des données déstructurées \\
            & & + Requêtage avec un langage proche du sql : N1QL \\
            & & + Avantage du monde relationnel et du monde NoSQL \\
            & & - Utilisé en production mais encore jeune \\
            \hline
        \end{tabular}
        \end{center}


    \subsection{Autres technologies}
    \label{subsec:autrestechno}

        \subsubsection{Elasticsearch}
        \label{subsubsec:elastic}
        ElasticSearch est une base de données orientée sur la recherche. Elle propose des possibilités de recherche intelligente.
        En effet, il est par exemple possible d'effectuer des recherches mal orthographiées, ElasticSearch se charge alors de trouver
        des ressemblances entre les mots et propose alors des résultats avec un taux de confiance.

        ElasticSearch est accessible par une API REST permettant d'effectuer des requêtes facilement. Elle est facilement scalable :
        la réplication des données au sein d'un cluster s'effectue en toute transparence.

        C'est donc une technologie qu'il serait intéressant d'utiliser dans le projet, pour permettre à l'utilisateur d'effectuer
        des recherches d'articles efficacement.

        Dans la plupart des cas d'usages, elle n'est pas utilisée en base de données principale. En général, les modifications
        de données sont effectuées sur une base de données principale (PostgreSQL, MongoDB, etc) puis les consultations/recherches
        de données sont effectuées sur ElasticSearch. Les données doivent donc être répliquées sur ElasticSearch.

        Cette séparation des tâches est intéressante d'un point de vue performance. En effet, en cas d'utilisation intensive
        des ressources du serveur, il serait tout à fait envisageable d'avoir une base de données principale sur un serveur,
        et ElasticSearch sur un autre.

        \subsubsection{OpenSeaDragon}
        \label{subsubsec:openseagdragon}
        OpenSeadragon est une visionneuse d’image haute résolution open source. C’est la visionneuse utilisée par le site
        “Chronicling America”. Le plugin est implémenté en Javascript pur et est adapté aux périphériques desktop et mobiles,
        ce qui convient à notre projet. (Pas besoin de framework tel que JQuery). Il est entièrement customisable, on peut
        y ajouter des overlays (polygones délimiteurs de contenus), boutons de navigation, etc. De plus son caractère open
        source permettrait son utilisation et sa modification dans le cas où il nous faille y ajouter des fonctionnalités.
        C’est un exemple de plugin Javascript que nous pourrions utiliser. Cependant, un cahier des charges plus poussé du
        projet nous obligera peut-être à nous tourner vers l’utilisation de plusieurs plugins et de coder un lot de fonctionnalités
        nous-mêmes à l’aide de JQuery.

        \subsubsection{Bootstrap}
        \label{subsubsec:bootstrap}
        Bootstrap est un framework HTML, CSS et Javascript, et c’est le framework le plus populaire du genre.
        Côté CSS, il permet de structurer un site grâce à un système de grilles , cela permet de créer un site
        totalement responsive, adapté aux mobiles, tablettes et ordinateurs de bureau. De plus, il fourni des
        éléments de design pour les éléments HTML simple tels que les formulaires, champs et boutons par exemple.
        Dans le cadre de notre projet, ce framework peut servir à rendre notre solution responsive. Cela permet
        de coder une seule application web fonctionnelle sur tous les supports au lieu de créer des applications
        pour téléphones mobiles.

    \subsection{Besoins de la base de données}
    \label{subsec:besoinplateforme}
    Les caractéristiques mises en avant précédemment sont d’ordre général. Nous allons à présent regarder
    vis-à-vis des particularités de notre base de données.

    Les données que nous manipulerons seront nombreuses, l’application est destinée, entre autres,
    à des services d’archives départementales qui ont des dizaines de milliers de documents numérises
    et ce nombre s’accroit. Ce sont aussi des données plutôt volumineuses (plusieurs Mo le document).
    Cela signifie que nous avons besoin d’une solution qui soit optimisée pour ce genre de cas tout
    en restant performant car l’utilisateur ne doit pas attendre l’affichage du document.
    Donc, MySQL ne figure pas dans les meilleures solutions pour cette particularité.

    Il faut prendre aussi une particularité intrinsèque au projet : la connaissance du langage.
    En effet, utiliser un langage que nous maîtrisons déjà nous permettrait de gagner du temps.
    C’est le cas du SQL que nous avons vu lors de notre cursus scolaire. Cependant apprendre un
    nouveau langage est toujours un apport ludique. Les projets sont l’occasion d’utiliser des
    langages différents de ceux enseignés dans le cursus. Mais il ne faut pas non plus choisir un
    langage trop obscur car le projet sera probablement poursuivi les années suivantes. Il serait
    préférable que le langage choisi soit motivant pour les prochains groupes qui travailleront sur le projet.

    Enfin, certaines solutions sont des licences propriétaires. Cela signifie que l’utiliser nécessite d’acheter
    une licence. Nous aimerions éviter d’avoir à utiliser ce genre de solutions non seulement pour une raison
    financière mais aussi parce que nous préférerions  que notre plate-forme soit libre, "free as in freedom"
    pour citer Mr Richard M. Stallman.
