\section{Organisation du projet}
	\subsection{Répartition des rôles}
	    Étant donné que quatre membres du groupe partent à l'étranger au second semestre, nous avons fixé une organisation précise pour que le projet se déroule de la meilleure manière possible. Les différents rôles ainsi décidés au sein du groupe sont décrits ci-dessous.
	    
	    \paragraph{Coordinateur} Le coordinateur s'occupe de planifier les réunions de projet et de les animer. Il est le contact privilégié des encadrants. Son rôle est aussi de s'assurer de l'avancée des rapports et de leur relecture. Il changera régulièrement afin que tout le groupe assure ce rôle. Hoel Kervadec est le premier coordinateur jusqu'à la livraison de ce rapport. Corentin Nicole sera le prochain coordinateur, puis Florent Mallard assurera ce rôle. De cette manière, chacun des étudiants partant en études à l'étranger aura tenu la place de coordinateur avant son départ.
	    
	    \paragraph{Administrateur système} L'administrateur système s'occupe de maintenir à jour les plates-formes et outils utilisés durant le projet (GitHub, GoogleDrive, etc.). Valentin Esmieu est en charge de ces plates-formes.
	    
	    \paragraph{Scribe} Un scribe est volontaire à chaque début de réunion afin de prendre des notes. Ce rôle est communément assuré par les personnes disposant de leur ordinateur à l'{\sc Insa} afin de rédiger un compte-rendu en direct sur le Google Drive.
	    
	    \paragraph{Responsable planification} Le responsable planification est en charge du suivi et de la mise à jour de la planification du projet. Celui-ci utilisera le logiciel MS Project, comme indiqué dans la Section \ref{sec:outils}.
	    
	    \paragraph{Décompte du temps} Au sein du groupe, le décompte du temps passé sur le projet se fait de manière autonome. Nous utilisons pour cela un tableur créé sur Google Drive. Le coordinateur et le responsable planification consulteront ce tableur afin d'effectuer au mieux la répartition des tâches restantes.

	    Nous avons également décidé de nous réunir de manière hebdomadaire le mercredi soir. Au cours de ces réunions, nous nous concertons sur les tâches à réaliser, et les répartissons entre nous. C'est aussi à ce moment que nous débattons sur les principales questions qui seront posées au cours de la prochaine réunion avec nos encadrants.

	    
	\subsection{Planification}
		
		En septembre, nous avons suivi des cours sur les ADTrees (donnés par Barbara Kordy) afin de se familiariser avec le concept, puis nous avons découvert ADTool et son fonctionnement. Nous avons ensuite jugé utile de suivre un cours de cryptographie afin de mieux saisir les protocoles de communication sécurisée. Gildas Avoine nous a donc dispensé un cours de deux heures, ce qui nous a permis d'appréhender les concepts de protection des cartes Korrigo. Ceux-ci pourraient nous être utiles pour analyser et valuer les risques liés à ce type de carte.
		
		Puis, en octobre, nous avons réalisé l'étude de l'existant sur le sujet de notre projet. Puis nous avons élaboré notre cahier des charges et défini l'architecture de notre logiciel. Enfin, nous nous sommes attelés à la rédaction de ce rapport de pré-étude. Comme nous avons noté le temps passé à travailler sur ce rapport, nous pouvons en déduire approximativement le temps qu'il nous faudra pour livrer les suivants. Pour rendre un rapport de vingt pages, nous avons passé, entre rédaction et correction, environ vingt-cinq heures chacun.
		
		Pour la suite, bien que la planification ne soit pas encore totalement établie, nous pouvons déjà en donner les principaux axes. 

		En novembre, nous allons définir les spécifications fonctionnelles de notre logiciel afin de préciser davantage les différentes interactions entre les modules présents. Nous nous réunirons pour détailler l'interface graphique, puis nous nous mettrons d'accord sur la répartition des tâches et la manière d'implémenter les améliorations d'ADTool.

		Ensuite, jusqu'aux vacances de Noël, nous établirons la planification complète du projet. Elle sera plus détaillée et contiendra un diagramme de Gantt expliquant la répartition des différentes tâches entreprises et leur ordonnancement. À ce moment, nous aurons donc tous les outils nécessaires à la modélisation de notre projet. Au terme de cette échéance, nous serons en mesure de présenter notre planification ainsi que nos spécifications lors des soutenances des 18 et 19 décembre.

		De début 2015 à mi-février, nous définirons l'architecture interne de notre logiciel, et nous pourrons en parallèle commencer l'implémentation. Suivra enfin le développement complet de notre logiciel, jusqu'à la fin de l'année scolaire. Enfin, nous présenterons le projet complet durant les soutenances des 28 et 29 mai.

	    
