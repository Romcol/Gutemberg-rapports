\section{Les cibles}
\label{sec:cibles}
    L’application vise ici la totalité des utilisateurs qui pourraient être intéressés pour accéder à la presse ancienne numérisée.
    On a donc un public potentiel de toute provenance, des sciences humaines aux sciences techniques, qui trouverait de l’utilité à
    découvrir d’anciens articles sur des thématiques particulière. On peut aussi imaginer des personnes de tout âge cherchant à
    utiliser la plateforme : il n’est pas invraisemblable de trouver un intérêt à la création de parcours thématiques pour des exposés
    scolaires, et on peut donc se retrouver avec des utilisateurs très jeunes qui accéderaient à la plateforme, mais aussi des personnes
    pluparts âgées, qui ont moins de facilité avec les technologies récentes, qui pourraient vouloir accéder à cette information.

    Il faut donc prendre en compte que l’on va avoir une base d’utilisateur très large, et que l’utilisation de la plateforme doit être
    le plus simple et intuitif possible afin que ceux-ci ne soient pas perdus ou découragés. De plus, une base d’utilisateurs venant
    de divers horizons signifie aussi des utilisateurs souhaitant accéder à l’application à travers diverses plateforme ; il ne sera pas
    étrange de retrouver des utilisateurs sous Windows comme sous Mac, Android, Linux… il sera donc nécessaire d’avoir une plateforme
    compatible sur n’importe quel distribution, pouvant s’adapter à n’importe quel format (PC, tablette mais aussi smartphone).
