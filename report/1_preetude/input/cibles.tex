\section{Les cibles}
\label{sec:cibles}
    L’application vise ici un large panel d’utilisateurs qui pourraient être intéressés
    par l’accès à la presse ancienne numérisée. On remarque qu’une grande partie
    des utilisateurs de services d’accès aux documents anciens font des recherches sur la généalogie.
    De plus, on peut imaginer que des personnes de tout âge cherchent à utiliser la plateforme :
    il y donc a un intérêt à la création de parcours thématiques pour des exposés, notamment scolaires.
    Des personnes plus âgées, ayant moins d’affinité avec les technologies récentes,
    qui peuvent vouloir accéder à cette information.

    Il faut donc prendre en compte que l’on va avoir une base d’utilisateurs très large,
    et que l’utilisation de la plateforme doit être le plus simple et intuitif possible
    afin que ceux-ci ne soient pas perdus ou découragés. De plus, une base d’utilisateurs
    venant de divers horizons signifie aussi des utilisateurs souhaitant accéder à l’application
    à travers diverses plateforme ; il ne sera pas étrange de retrouver des utilisateurs
    sous Windows comme sous Linux, Mac, Android, ... il sera donc nécessaire d’avoir une plateforme
    compatible avec n’importe quelle plateforme (PC, tablette mais aussi smartphone).
