\section{Le projet, le besoin}
\label{sec:intro}

    \subsection{Contexte du projet}
    \label{subsec:contexte}
    Ce projet qui s’inscrit dans le cadre de la formation de 4ème année à l’INSA de Rennes a pour but
    de former les étudiants à un travail d’ingénierie de groupe sur une réalisation technique de taille conséquente.
    Celle-ci permettra de mettre en pratique les connaissances en génie logiciel mais aussi en gestion
    de projet et ce afin de mener, à travers toutes les phases de conception et de réalisation, un véritable projet.

    Notre projet est proposé par l’équipe Intuidoc de l’Irisa, qui travaille depuis de nombreuses années
    avec les archives départementales de plusieurs départements pour améliorer l’accès par le contenu
    aux images numérisées de documents anciens conservés aux archives. Dans ce contexte précis,
    il est nécessaire de mettre à disposition du public un système d’accès permettant d’accéder à cette information.

    Les travaux de cette équipe ont permis de construire un prototype qui est capable d’interpréter des images
    de journaux anciens du XIXème et du XXème siècle pour produire une représentation XML de l’ensemble du contenu
    de chaque page. Et c’est, en s’appuyant sur ce prototype, que nous devrons réaliser une plateforme de navigation
    de documents numérisés ; nous disposerons d’un certain panel de documents déjà numérisés par l’équipe afin
    d’effectuer des tests sur la plateforme. Ce projet est également réalisé en coopération avec les Archives départementales
    d’Ille-et-Vilaine, qui pourra par ailleurs nous fournir de nouveaux documents sur certaines thématiques ou certaines
    dates afin d’élargire panel de tests.

    De plus, Mr. Yoann Royer, chef de projet chez Sopra Steria et anciennement diplômé de l’INSA de Rennes,
    nous accompagnera durant ce projet. Cette accompagnement a lieu suite à l’initiative de l’INSA de mettre
    en relation des professionnels avec les étudiants pour échanger sur l’organisation du projet,
    la création du planning, le déroulement de celui-ci... et d’obtenir aussi le point de vue d’une personne extérieure au projet.


    \subsection{Besoin à l'origine du projet}
    \label{subsec:besoin}
    Plusieurs millions de journaux datant de la révolution française jusqu’à aujourd’hui sont actuellement
    présents dans les archives. De manière générale, l’accès aux documents d’archives est améliorable.
    Même si cela concerne moins les journaux anciens, qui sont en bien meilleur état que des fiches
    d’état civil ou tout type de documents ayant été déplacés un certain nombre de fois avant d’être archivés,
    de nombreux documents ont été endommagés avant d’arriver aux archives. Même si ceux-ci sont bien conservés,
    lorsque l’on souhaite accéder à ces documents, on les manipulent et ils peuvent se déteriorer.
    De plus, certaines archives contiennent des kilométrages de rayons et, même si dans le cadre de la presse ancienne
    le nombre de documents stockés est moindre, il n’en reste que stocker de manière physique autant
    de documents pose plus de difficultés à indexer, conserver et stocker que des documents numériques.
    Par ailleurs, quelqu’un souhaitant consulter un document particulier va devoir se rendre à l’archive ou va devoir
    faire une demande pour en obtenir une copie, il ne peut pas y accéder instantanément. Il devient donc nécessaire,
    dans une époque où tout a tendance à devenir accessible par internet, de trouver une nouvelle façon de stocker
    ces documents qui pourrait permettre un échange plus simple et étendu.

    \pagebreak

    Cependant numériser ces documents ne résoudra pas le problème de l’accessibilité de ces derniers. Il est certes
    plus aisé de stocker et rechercher ceux-ci sur un disque dur que dans un entrepôt physique, cependant il n’en
    reste pas moins qu’il reste difficile de pouvoir rechercher efficacement certains documents suivant des besoins précis.

    C’est donc dans ce contexte particulier que se place notre projet. Nous nous plaçons, ici, dans le cas
    de la presse ancienne, et donc dans la gestion de version numérisée de journaux pouvant dater
    du XIXème siècle; mais on peut imaginer que le résultat final puisse aussi permettre de gérer
    des documents numérisés de provenances diverses. On a besoin de disposer d’une plateforme
    qui permettra de répertorier tous ces documents qui ont – et seront – numérisés par les archives
    dans une idée de partage, de facilité et de libre accès de l’information.
