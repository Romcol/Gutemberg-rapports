\section{Le projet, le besoin}
\label{sec:intro}

    \subsection{Contexte du projet}
    \label{subsec:contexte}
    Ce projet, qui s’inscrit dans le cadre de la formation de 4ème année à l’INSA de Rennes, a pour but de former les étudiants à un travail d’ingénierie de groupe sur une réalisation technique de taille conséquente. Celle-ci permettra de mettre en pratique les connaissances en génie logiciel mais aussi en gestion de projet et ce afin de mener à travers toutes les phases de conception et de réalisation un véritable projet.

    Notre projet est proposé par l’équipe Intuidoc de l’Irisa, qui travaille depuis de nombreuses années avec les archives départementales de plusieurs départements pour améliorer l’accès par le contenu aux images numérisées de documents anciens conservés aux archives. Dans ce contexte précis, il est nécessaire de mettre à disposition du public un moyen permettant d’accéder à cette information.

    Les travaux de cette équipe ont permis de construire un prototype qui est capable d’interpréter des images de journaux anciens du XIXème et du XXème siècle pour produire une représentation XML de l’ensemble du contenu de chaque page. Et c’est en s’appuyant sur ce prototype que nous devrons réaliser une plateforme de navigation dans des documents numérisés ; nous disposerons d’un certain panel de documents déjà numérisés par l’équipe afin d’effectuer des tests sur la plateforme. Ce projet est également réalisé en coopération avec les Archives départementales d’Ille-et-Vilaine, qui pourra par ailleurs nous fournir de nouveaux documents sur certaines thématiques ou certaines dates afin d’avoir un plus large panel de tests.

    \subsection{Besoin à l'origine du projet}
    \label{subsec:besoin}
    C’est donc dans ce contexte particulier que se place notre projet. On se place ici dans le cas de la presse ancienne, et donc dans la gestion de version numérisée de journaux pouvant dater du XIXème siècle, mais on peut imaginer que le résultat final puisse aussi permettre de gérer des documents numérisés de provenances diverses. On a besoin de disposer d’une plateforme qui permettra de répertorier tous ces documents qui ont – et seront – numérisés par les archives dans une idée de partage, de facilité et de libre accès de l’information. 

    Plusieurs millions de journaux datant de la révolution française jusqu’à aujourd’hui sont actuellement présents dans les archives. Actuellement l’accès aux documents est améliorable, car ceux-ci sont stockés sur des kilomètres de rayons et pouvoir accéder et lire le contenu d’un document particulier, bien que bien indexé, reste une tâche qui n’est pas aisé. Il est aussi obligatoire de devoir se rendre à cette archive si l’on souhaite lire un document. Il est nécessaire dans une époque où tout devient accessible par internet d’imaginer une nouvelle façon de stocker ces documents qui permettrait un échange plus simple et étendu. De plus, de nombreux documents ont été endommagés avant d’arriver aux archives, et même si ceux-ci sont bien conservés des accès répétés à ceux-ci peuvent risquer de les endommager. Il est d’autant plus nécessaire de les sauvegarder avant que ceux-ci ne deviennent inexploitables.

    Cependant, numériser ces documents ne résoudra pas le problème de l’accessibilité de ceux-ci. Il est certes plus aisé de stocker et rechercher ceux-ci sur un disque dur que dans un entrepôt physique, cependant il n’en reste pas moins qu’il reste difficile de pouvoir recherche efficacement certains documents suivant des besoins précis.

