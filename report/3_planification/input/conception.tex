\section{Phase de conception}

    La phase de conception démarrera à l'issue de notre étude fonctionnelle qui s'achèvera lors des soutenances de projet le 18 décembre.  8 semaines y seront dédiées, qui aboutiront à un rapport de conception à rendre le 5 février dans sa première version, le 11 février dans sa version finale. La création de ce rapport sera divisée en trois tâches (total : 18h) :
\begin{itemize}
\item L'architecture global et communication entre les différents modules : client, serveur d'application, base de données principale, base de données de recherche : 4h
\item La modélisation UML du schéma de la base de données : 6h
\item Redaction du rapport : 8h
\end{itemize}

    En parallèle du travail de rédaction du rapport de conception, il sera primordial de se préparer à la phase de développement mettant en place un environnement de développement (total : 18h) :
\begin{itemize}
\item mise en place d'un serveur de déploiement (total : 6h) :
    \begin{itemize}
    \item configuration serveur : serveur HTTP, utilisateurs, accès ssh, ouvertures de ports, etc : 3h
    \item configuration du déploiement avec git : 1h
    \item installation + configuration ElasticSearch : 1h
    \item installation + configuration \[BASE_DE_DONNEES\] : 1h
    \end{itemize}
\item installation des outils sur les postes de travail : 4h/developpeur (total : 12h).
\end{itemize}

De plus, avant de démarrer la phase de développement, il est nécessaire que chacun des développeurs s'autoforme aux technologies qui seront utilisées (total : 24h) :
\begin{itemize}
\item Elasticsearch : 2h/developpeur soit 6h
\item \[BASE_DE_DONNEES\] : 2h/developpeur soit 6h
\item \[FRAMEWORK\] : 4h/developpeur soit 12h
\end{itemize}

    L'autoformation sur les technologies front-end (Bootstrap, OpenSeadragon) sera considérée comme incluse dans les tâches de développement associées. En effet, ces technologies ne demandant pas de compréhensions de notions complexes, l'autoformation pourra s'effectuer au fur et a mesure des besoins lors du développement.
    De même, le temps estimée pour l'autoformation sur les bases de données et frameworks ne permettra pas la maitrîse totale de ceux-ci, mais l'idée est d'être initié afin de commencer la phase de développement dans les meilleures conditions.

Pour résumer, nous aurons les taches suivantes (total 60h) :
\begin{itemize}
\item Conception et redaction du rapport : 18h
\item Mise en place d'un environnement de développement (serveur + postes de travail) : 18h
\item Autoformation : 24h
\end{itemize}

