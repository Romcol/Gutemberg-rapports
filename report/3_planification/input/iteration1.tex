\section{Itération 1 : moteur de recherche}

Le moteur de recherche est l'une des principales fonctionnalités de la plateforme. Il permettra à l'utilisateur de rechercher dans la base de données à l'aide de différents critères.

\textbf{La recherche :} la première étape de la mise en place de la recherche est l'installation d'Elasticsearch. C'est lui qui fera la recherche au sein de la base. Cependant, il est nécessaire qu'il puisse communiquer avec la base de donnée afin d'en récupérer le contenu. Elasticsearch dispose d'ailleurs d'un plugin qui lui permet de se lier à une base de donner et de mettre à jour son index lors d'ajouts ou de modifications de la base. Elasticsearch sera alors capable de gérer une recherche d'articles, de journaux ou de revues de presse. Cette étape est la principale de la tâche et prendra 10h. Ensuite, il faudra implémenter la communication entre Elasticsearch et l'interface utilisateur. C'est-à-dire une recherche faite par l'utilisateur sera transmise à Elasticsearch et les résultats seront retournés à l'utilisateur. Cette étape prendra 4h.

\textbf{L'affichage des résultats :} les résultats de la recherche ne consistent uniquement en une liste d'articles. Des informations sont affichées pour chaque articles telles que le titre du journal, la date, s'il fait partie d'une revue de presse... Pour afficher ces informations, il faudra questionner la base de données pour obtenir ces informations ou les demander à Elasticsearch. 8h seront nécessaire pour résaliser cette étape. Puis il faudra les afficher dans les résultats, cela prendra 4h. Pour les recherches de journaux et de revues de presse, il faudra faire de même. L'essentiel étant déjà fait dans les étapes précédentes, celle-ci ne prendra pas plus de 4h. 

\textbf{La gestion des critères et des filtres :} Elasticsearch est capable de gérer les recherches avec des critères ainsi que de trier les résultats obtenus.  Il suffira donc d'implémenter l'interface utilisateur pour l'utilisation de ces critères et d'envoyer ceux qui ont été choisis à Elasticsearch. Pour les filtres, c'est le même procédé. Les deux étapes prendront 3h chacune.