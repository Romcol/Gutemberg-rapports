\section{Phase de développement}
\label{sec:dev}

	Afin d'obtenir des retours sur l'application et d'adapter notre développement ou de corriger certaines fonctionnalités, nous avons donc décidé de découper la phase de développement en différentes itérations pour réaliser plusieurs livrables. Ces livraisons précéderont une série de tests d'intégration pour vérifier et corriger la présence de bugs, et un temps de développement est prévu afin de prendre en compte les remarques et les retours sur la version livrée et adapter l'application. Les tests unitaires, réalisés pour vérifier le fonctionnement de chaque tâche, ne sont pas détaillés et sont compris dans les durées des tâches décrites par la suite, vous pouvez cependant retrouver le détail de ceux-ci dans le diagramme de Gantt à la fin du rapport. Nous avons pris en considération des durées de test allant de 50\% à 100\% de la durée de la tâche, suivant la complexité de celle-ci.

	Nous avons décidé de découper le projet en trois itérations. Les deux premières représentent les deux grosses fonctionnalités primordiales, mais aussi les plus complexes, à réaliser. Lors de la première itération, nous fournirons le moteur de recherche de celle-ci, ainsi qu'une base de données contenant des jeux de tests. Cela permettra de vérifier le bon fonctionnement de la première partie du site, et de confirmer qu'elle correspond bien à ce qui est attendu. La version de la seconde itération, dans la continuité de la première, contiendra en plus la page de consultation d'un document, avec toutes les fonctionnalités liées. Cette page étant la partie centrale du site, il est nécessaire de rendre celle-ci en milieu de phase de développement pour avoir le temps d'y apporter des modifications. Enfin, la dernière itération contiendra toutes les fonctionnalités annexes; c'est-à-dire la gestion d'utilisateurs, la gestion des revues de presse et la page d'accueil. Celle-ci sera rendue peu de temps avant la livraison finale, pour permettre d'apporter quelques dernières modifications si le besoin s'en fait ressentir.

	En résumé, voici comment se déroulera la phase de développement :

	\begin{figure}[H]
        \centering
        \includegraphics[width=\textwidth]{figure/schema_developpement.png}
            \caption{Schéma représentant la phase de développement}
            \label{fig:sch_dev}
    \end{figure}