\section{Rappel des fonctionnalités}

	Nous avions découpé la plateforme en 5 'thèmes'. Chacun de ces thèmes contient de multiples fonctionnalités, qui donneront naissance à de multiples tâches, et qui se retrouvent être nos 5 itérations. Voici un résumé de ceux-ci, que vous pouvez retrouver dans le rapport précédent. Si vous avez déjà lu le rapport précédent, et n'avez pas besoin d'un rappel sur celles-ci, il est possible de directement passer à la partie suivante (lien à faire).

	\textbf{Moteur de recherche :} L'utilisateur aura accès à une page qui va lui permettre d'effectuer une recherche dans la base de données. Il peut rechercher des journaux, des articles, et des revues de presse, chaque type de document ayant ses propres filtres de recherche. En cherchant un \textbf{journal}, l'utilisateur peut chercher celui-ci en fonction de son nom et de sa date d'édition. Les résultats affichés contiendront le nom du journal et la date de parution. Cliquer sur un résultat l'amènera sur la page de consultation du journal. En cherchant un \textbf{article}, l'utilisateur peut effectuer sa recherche sur le titre, le contenu, l'auteur, les tags associés, et sur les filtres pour un journal. Les résultats affichés contiendront le titre de l'article, l'auteur, la date de parution, le journal auquel il appartient, les tags qui lui sont associés, et un aperçu du contenu. Il sera aussi indiqué si ceux-ci appartiennent à une revue de presse. Si une recherche sur le contenu était effectuée, le texte trouvé sera mis en évidence. Sinon les premières lignes de l'article seront affichées. En cherchant une \textbf{revue de presse}, l'utilisateur peut effectuer sa recherche sur le nom de celle-ci, et sur sa description. Les résultats affichés contiendront le nom de la revue de presse, sa description, et ses 3 premiers articles. Il est possible de les trier suivant leur date de création ou de modification. En sélectionnant un résultat, l'utilisateur arrive sur la page de la revue de presse. 

	\textbf{Consultation de document :} En accédant à un document, l'utilisateur arrive sur une page de consultation. Celle-ci est décomposée en 3 parties. La partie gauche est consacrée à l'information. Elle contient les informations sur le journal en cours de lecture (son nom et sa date d'édition), la page de journal en train d'être consultée (liste des articles et des images) lui permettant de mettre en évidence l'article, et l'article en cours de lecture (titre, tags) si un article est sélectionné. Si un article est sélectionné, cette partie lui permettra d'ajouter des tags (à la manière de Twitter) ou d'ajouter l'article à une revue de presse; un champ de recherche sera accessible pour trouver la revue de presse auquel il souhaite l'ajouter; celles auxquelles il a contribué apparaissant en premier. La partie centrale de la page est consacrée à la consultation. 

	\textbf{Revue de presse}

	\textbf{Page d'accueil}

	\textbf{Gestion d'utilisateur}