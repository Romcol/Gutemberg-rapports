\section{Rappel des fonctionnalités}

	Nous avions découpé la plateforme en 5 'thèmes'. Chacun de ces thèmes contient de multiples fonctionnalités, qui donneront naissance à de multiples tâches, et qui se retrouvent être nos 5 itérations. Voici un résumé de ceux-ci, que vous pouvez retrouver dans le rapport précédent. Si vous avez déjà lu le rapport précédent, et n'avez pas besoin d'un rappel sur celles-ci, il est possible de directement passer à la partie suivante (lien à faire).

	\textbf{Moteur de recherche :} L'utilisateur aura accès à une page qui va lui permettre d'effectuer une recherche dans la base de données. Il peut rechercher des journaux, des articles, et des revues de presse, chaque type de document ayant ses propres filtres de recherche. En cherchant un \textbf{journal}, l'utilisateur peut chercher celui-ci en fonction de son nom et de sa date d'édition. Les résultats affichés contiendront le nom du journal et la date de parution. Cliquer sur un résultat l'amènera sur la page de consultation du journal. En cherchant un \textbf{article}, l'utilisateur peut effectuer sa recherche sur le titre, le contenu, l'auteur, les tags associés, et sur les filtres pour un journal. Les résultats affichés contiendront le titre de l'article, l'auteur, la date de parution, le journal auquel il appartient, les tags qui lui sont associés, et un aperçu du contenu. Il sera aussi indiqué si ceux-ci appartiennent à une revue de presse. Si une recherche sur le contenu était effectuée, le texte trouvé sera mis en évidence. Sinon les premières lignes de l'article seront affichées. En cherchant une \textbf{revue de presse}, l'utilisateur peut effectuer sa recherche sur le nom de celle-ci, et sur sa description. Les résultats affichés contiendront le nom de la revue de presse, sa description, et ses 3 premiers articles. Il est possible de les trier suivant leur date de création ou de modification. En sélectionnant un résultat, l'utilisateur arrive sur la page de la revue de presse. 

	\textbf{Consultation de document :} En accédant à un document, l'utilisateur arrive sur une page de consultation. Celle-ci est décomposée en 3 parties. La partie gauche est consacrée à l'information. Elle contient les informations sur le journal en cours de lecture (son nom et sa date d'édition), la page de journal en train d'être consultée (liste des articles et des images) lui permettant de mettre en évidence l'article, et l'article en cours de lecture (titre, tags) si un article est sélectionné. Si un article est sélectionné, cette partie lui permettra d'ajouter des tags (à la manière de Twitter) ou d'ajouter l'article à une revue de presse; un champ de recherche sera accessible pour trouver la revue de presse auquel il souhaite l'ajouter; celles auxquelles il a contribué apparaissant en premier. Il pourra aussi ajouter l'article à ses favoris. La partie centrale de la page est consacrée à la consultation. Sur cette partie est présente une visionneuse, qui contiendra la page en cours de lecture, et des boutons permettant de naviguer dans le journal (page suivante/précédente) ou, lorsqu'on arrive sur cette page par une revue de presse, dans la revue de presse (article précédent/suivant). Sur la visionneuse, il est possible de zoomer/dézoomer et de se déplacer sur la page. Lors de l'arrivée sur la page de consultation, tous les articles seront mis en évidence à l'aide de calque transparent de couleur différente. Par la suite, lorsqu'un article est sélectionné, ces calques disparaîtront pour avoir un unique calque autour de celui sélectionné. Un champ de recherche au dessus de la visionneuse est disponible pour chercher des mots dans la page. Dans ce cas, les mots trouvés seront eux aussi mis en évidence à l'aide de calques. Enfin la dernière partie sert à guider l'utilisateur. Sur cette partie, trois parties peuvent être présentes suivant le contexte. Tout d'abord, les articles similaires, lorsqu'un article est sélectionné. Cliquer sur un de ces articles le redirigera vers la page de consultation de celui-ci. Ensuite, si l'article appartient à des revues de presses, une liste de celles associées sera disponible. Cliquer sur une des revues emmènera l'utilisateur sur la page de celle-ci. Enfin, si l'utilisateur est en train d'en consulter une, la liste de tous les articles de la revue de presse sera affichée; comme pour les articles similaires, il peut accéder à un de ceux-ci.

	\textbf{Revue de presse :} Chaque revue de presse, une fois créé, peut-être consultée. Accéder à la page d'une revue de presse permet de voir son nom, sa description et tous les articles associés à celle-ci, dans l'ordre. Si l'utilisateur est aussi le créateur de la revue de presse, il a la possibilité de changer l'ordre des articles de la revue de presse, mais aussi de supprimer des articles de celle-ci. Il est aussi possible pour tous les utilisateurs de naviguer dans la revue de presse; qui les amènerons sur la page de consultation du premier article, avec les boutons permettant de naviguer entre les articles de celle-ci.

	\textbf{Page d'accueil :} En arrivant sur la plateforme, l'utilisateur arrive sur une page d'accueil. Il peut, à partir de cette page, accéder à la page de recherche ou à son profil utilisateur. Il peut aussi effectuer directement une recherche sur le nom du document qu'il recherche (journal, article, revue de presse). Cette recherche l'amènera sur la page de recherche, avec un premier résultat. Sur la page d'accueil, il lui sera aussi possible d'accéder directement aux revues de presse récemment créées, à celles récemment modifiées mais aussi aux articles les plus consultés.

	\textbf{Gestion d'utilisateur :} L'utilisateur dispose d'un profil utilisateur. Trois listes sont présentes sur cette page; les revues de presse qu'il a créé, les revues de presse auquelles il a contribué et les articles qu'il a ajouté en favoris. En cliquant sur une des revues de presse, il accède directement à la page de celle-ci. Il peut supprimer un article de ses favoris, ou cliquer dessus pour accéder directement sur la page de consultation de celui-ci. C'est aussi à partir de cette page qu'il peut créer une nouvelle revue de presse. Sur cette page, il peut aussi accéder à ses paramètres pour modifier ses informations, à savoir son adresse mail et son mot de passe.