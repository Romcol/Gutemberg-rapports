\section{Itération 2 : Consultation de documents}

	Après avoir développé le moteur de recherche, nous entrerons dans la mise en place des fonctionnalités liées à la page de consultation de documents. Comme nous l'avions énoncé dans le rapport de spécifications, trois modes de consultation seront proposés; article, journal et revue de presse. Trois objectifs sont définis pour cette consultation; consulter, informer et guider le lecteur.

	Tout d'abord, nous allons mettre en place une visionneuse de documents, Openseadragon. C'est une visionneuse d'image haute définition adaptée à notre projet dans la mesure où elle permet d'ajouter des overlays pour de nouvelles fonctionnalités (boutons de navigation, polygones délémiteurs, zoom, calques, ...). Cette étape de développement et de tests est estimée à 24h.

	Ensuite nous rajouterons une fonctionnalités de recherche dans le document à la visionneuse (5h).

	Puis vient l'implémentation des fonctionnalités d'information. Il est question ici :
	Dans un premier temps, de développer une interface permettant à l'utilisateur de rajouter des informations au document. Il s'agit notamment de :
		\begin{itemize}
		\item Gestion des tags (10h)
		\item ajout à une revue de presse ou aux favoris (10h)
		\end{itemize}

	Dans un second temps, il est question de réaliser une architecture du document : interroger la base de données afin de revueillir toutes les informations importantes liées au document consulté. Les informations affichées pourront être par exemple le titre, la date d'un journal, les articles associés, le titre d'un aticle, les tags... Cette tâche nécessitera 8h de travail.

	En dernier lieu, nous allons développer les fonctionnalités de guidage.
	Parcours d'une revue de presse (8h)
	Proposition d'articles similaires, des revues de presse de même thématique (10h)

	À l'issue du développement de toutes ces fonctionnalités, nous réaliserons des test d'intégration estimés à 20h de travail.