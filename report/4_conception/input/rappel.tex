\section{Rappel de la phase d'analyse}
\label{sec:rappel}

	Nous avions, pendant cette phase, rédigé trois rapports présentant les étapes de l'analyse énoncées à l'introduction (la pré-étude\cite{Pretude}, la spécification\cite{Specs} et le planification\cite{Planif}). Nous allons, dans cette partie, résumer le rapport de spécifications fonctionnelles et celui de planification. La planification est  divisée en trois itérations dans lesquelles nous avons réparti les fonctionnalités de l'application : 

	\textbf{Moteur de recherche \textit{(Itération 1)}:} L'utilisateur a accès à une page où il peut rechercher (dans la base de données) des journaux, des articles, et des revues de presse ; chaque type de document ayant ses propres filtres de recherche. 
	En cherchant un \textbf{journal}, l'utilisateur peut chercher celui-ci en fonction de son nom et de sa date de parution. Cliquer sur un des résultats obtenus amènera l'utilisateur sur la page de consultation du journal. 
	En cherchant un \textbf{article}, l'utilisateur peut effectuer sa recherche sur le titre, le contenu, l'auteur et les tags associés. De même, il peut filtrer les résultats par journaux. Les résultats affichés contiendront les caractéristiques de l'article (titre, date, journal, aperçu du contenu, ...). Il sera aussi indiqué si l'article appartient à des revues de presse. 
	En cas de recherche sur le contenu, un extrait contenant le(s) terme(s) recherché(s) sera affiché. Sinon ce seront les premières lignes de l'article qui seront affichées. En cherchant une \textbf{revue de presse}, l'utilisateur peut effectuer sa recherche sur le nom de celle-ci et sur sa description. Les résultats affichés contiendront le nom de la revue de presse, sa description, et les trois premiers articles. Il est possible de les trier suivant leur date de création ou de modification. En sélectionnant un résultat, l'utilisateur arrive sur la page de la revue de presse. 

	\textbf{Consultation de document \textit{(Itération 2)}:} En accédant à  un document, l'utilisateur arrive sur une page de consultation. Celle-ci est décomposée en trois parties : l'information, la consultation et le guidage. La partie informative contient les informations sur le journal (nom, date), la page de journal en train d'être consultée (liste des articles et des images). Si un article est sélectionné, ses informations sont aussi affichées (titre, tags). De plus, cette partie lui permettra d'ajouter des tags (à  la manière de Twitter) ou d'ajouter l'article à une revue de presse. Pour cela, un champ de recherche sera accessible pour trouver la revue de presse auquel l'utilisateur souhaite l'ajouter ; celles auxquelles il a contribué apparaissant en premier. Il pourra aussi ajouter l'article à ses favoris. 

	La partie centrale de la page est consacrée à la consultation. Sur cette partie est présente une visionneuse, qui contiendra la page en cours de lecture. Des boutons permettront aussi de naviguer dans le journal (page suivante/précédente) et si l'utilisateur arrive sur cette page depuis une revue de presse, de naviguer dans la revue de presse (article précédent/suivant). Sur la visionneuse, il est possible de zoomer/dézoomer et de se déplacer sur la page. Lors de l'arrivée sur la page de consultation, tous les articles seront mis en évidence à  l'aide de calque transparent de couleur différente. Par la suite, lorsqu'un article est sélectionné, ces calques disparaîtront pour ne garder qu'un unique calque autour de celui sélectionné. Un champ de recherche au-dessus de la visionneuse est disponible pour chercher des mots dans la page. Dans ce cas, les mots trouvés seront eux aussi mis en évidence à l'aide de calques. 
	
	Enfin la dernière partie sert à guider l'utilisateur. Sur cette partie, trois sections peuvent être présentes suivant le contexte. Tout d'abord, les articles similaires, lorsqu'un article est sélectionné. Cliquer sur un de ces articles le redirigera vers la page de consultation de l'article. Ensuite, si l'article appartient à des revues de presses, une liste de celles-ci sera disponible. Enfin, si l'utilisateur est en train de consulter une revue de presse, la liste de tous les articles de celle-ci sera affichée.


	\textbf{Revue de presse \textit{(Itération 3)}:} Chaque revue de presse, une fois créé, peut être consultée. Accéder à la page d'une revue de presse permet de voir son nom, sa description et tous les articles associés à celle-ci. Si l'utilisateur est aussi le créateur de la revue de presse, il a la possibilité de changer l'ordre des articles de la revue de presse, mais aussi de supprimer des articles. Il est aussi possible pour tous les utilisateurs de naviguer dans la revue de presse et d'ajouter un article.

	\textbf{Page d'accueil \textit{(Itération 3)}:} En arrivant sur la plateforme, l'utilisateur arrive sur une page d'accueil. Il peut, à partir de cette page, effectuer une recherche ou accéder à son profil utilisateur. La recherche l'amènera sur la page de recherche, avec les résultats de la requête. Sur la page d'accueil, il lui sera aussi possible d'accéder directement aux revues de presse récemment créées, modifiées mais aussi aux articles les plus consultés.

	\textbf{Gestion d'utilisateur \textit{(Itération 3)}:} L'utilisateur dispose d'un profil utilisateur. Trois listes sont présentes sur cette page; les revues de presse qu'il a créées, les revues de presse auxquelles il a contribué et les articles qu'il a ajoutés en favoris. En cliquant sur une des revues de presse, il accède directement à  la page de celle-ci. Il peut supprimer un article de ses favoris, ou en sélectionner un pour le consulter. C'est aussi à partir de cette page qu'il peut créer une nouvelle revue de presse. Enfin, il peut accéder à ses paramètres pour modifier ses informations, à savoir son adresse mail et son mot de passe.

Cette spécification détaillée des fonctionnalités de la plate-forme nous a permis de choisir des technologies et ainsi d'établir une architecture du projet.
