\newpage
\section{Architecture de la base de données}
\label{sec:bdd}

\subsection{Origine de nos données}
\label{subsec:dataOrigin}

Comme nous l'avons dit dans l'introduction, nos données sont issus de fichiers XML que nous fournit l’équipe Intuidoc. Dans chaque fichier se trouve le contenu d'une page de journal. Pour récupérer ces données, nous avons réalisé un programme \textit{parser}. Ce programme prend les données qui nous intéresse et les insère directement dans la base de données MongoDB si elle est accessible. Il peut aussi créer des fichiers JSON qui contiennent les données pour pouvoir les importer dans une base de données. Le programme est en PHP car c'est un langage qui permet de lire le contenu de fichiers XML facilement.

\subsection{Base de données}
\label{subsec:mongo}

Notre base de données est gérée par un système NoSQL : MongoDB. Nos données sont reparties par collections (c’est l’équivalent d’une table en SQL). Dans ces collections, les données sont écrites sous forme de document JSON. Le format JSON permet de représenter l’information de manière structurée. Un document JSON est composé de paires nom/valeurs ou de listes ordonnées de valeurs. Ces valeurs peuvent être des objets eux-mêmes représentés sous format JSON ; des tableaux ou des valeurs génériques : nombre, chaîne de caractère, booléen. Nous utilisons la version 3.0 qui était la dernière lorsque nous avons commencé à étudier cette technologie.  


Voilà un exemple de document que nous aurons dans la collection Revues : 


\begin{verbatimtab}[3]
{
	‘_id’ : 1,
	‘Nom’ : ‘Mai 68’,
	‘Description’ : ‘Ceci est une revue de presse sur Mai 68’,
	‘TitreArticles’ : [ 
		{ ‘idArticle’ : ‘34354’, ‘Titre’ : ‘Effervescence dans les universités françaises’ },
		{ ‘idArticle’ : ‘78e64a’,  ‘Titre’ :	‘Premières barricades de Mai 68’ }
  ]
}
\end{verbatimtab}

Chaque document dispose d’un id afin de pouvoir le retrouver rapidement. Cela permet aussi de pouvoir référencer d’autres objets de la base de données. 

Notre base de données est composée de cinq collections :
\begin{itemize}
	\item \textbf{Article :} elle contient les informations de chaque articles : son titre, le titre du journal, la date. Pour pouvoir mettre l'article en surbrillance, nous stockons aussi les coordonnées de l'article sur la page. Nous faisons de même pour le titre dans le cas d'une recherche parmi les  titres. Quant au contenu de l'article, les mots sont stockés avec leurs coordonnées afin de pouvoir, eux aussi, les surligner s'ils correspondent aux termes recherchés. Enfin, la liste des revues contenant l'article et la liste des tags sont stockées ici.
	\item \textbf{Page :} les informations de chaque page sont stockées ici (titre, date, numéro de page, page précédente et suivante, nom de l'image). De plus, la liste des articles (titre et coordonnées) ainsi que la liste des légendes des images sont aussi stockées car ces informations sont affichées lorsqu'on visualise une page.
	\item \textbf{Revue :} les revues de presse sont stockées dans cette collection. Nous stockons le nom, la description et la date de création de la revue ainsi que la liste des ses articles avec leur titre. Ces derniers sont en effet affichés sur la page de la revue.
	\item \textbf{Utilisateur :} ici sont stockés le mail et le mot de passe crypté de chaque utilisateur. Nous y conservons aussi les listes des revues contribuées, créées avec leur titre et leur description ainsi que la liste des articles favoris avec leur titre, journal, date et un extrait.
	\end{itemize}


Le diagramme de la figure \ref{modelbdd} résume la modélisation de notre base de données.


\begin{figure}[H]
        \centering
        \includegraphics[width=\textwidth]{figure/ModelBDD.png}
            \caption{Modélisation de la base de données MongoDB}
            \label{modelbdd}
\end{figure}

Certaines informations peuvent sembler répétitives comme la liste des articles favoris qui contient les mêmes informations que celles contenues dans les documents de la collection Article. Cependant, avec MongoDB, nous ne pouvons pas faire de jointure entre les tables. Si nous voulons les informations sur l’article, il sera nécessaire de faire une deuxième requête dans la base avec l’id du document de l’article pour obtenir ces informations. C’est pourquoi certaines informations sont dupliquées, notamment celles qui seront affichées. Cela permet d'économiser des requêtes à la base de données et donc du temps. Cependant, étant donné que nous faisons les requêtes de recherche avec Elasticsearch\ref{subsec:elastic}, cela rend les requêtes très rapides. Par conséquent, il est possible que nous ne perdions pas de temps à faire des requêtes pour récupérer certaines données au lieu d'utiliser les données dupliquées. La duplication serait donc facultative et nous pourrions ainsi économiser en quantité de données stockées.


Chaque flèche représente une référence, c’est-à-dire que l’id du document correspondant est stockée afin de pouvoir le retrouver dans la base. 

\subsection{Elasticsearch}
\label{subsec:elastic}

Elasticsearch est un moteur de recherche, il nous servira pour rechercher parmi les articles de notre base, mais aussi pour n'importe quelle requête de recherche que nous aurons besoin de faire. Elasticsearch construit un index à partir des données afin de les parcourir rapidement et de manière optimisée. Pour notre projet, Elasticsearch indexera chaque collection de la base de données. Cela nous dispense de construire nous-mêmes l'index de la base de données et les requêtes de recherche seront plus rapides que si nous les envoyions directement à MongoDB. Nous utilisons la version 1.4.4 d'Elasticsearch. Une version plus récente existe (la version 2.1) mais le plugin que nous utilisons pour faire communiquer MongoDB et Elasticsearch, ne fonctionne pas sur cette version plus récente.


Habituellement, Elasticsearch dispose de sa propre base de données qu’il indexe. Nous utilisons donc le plugin elasticsearch-river-mongodb\cite{GitRiver}, plus précisément sa version la plus récente. Il permet à Elasticsearch d’utiliser une base de données MongoDB comme source de données. 


De plus, à chaque ajout, mise à jour ou suppression de données dans la base MongoDB, Elasticsearch met automatiquement à jour son index. Pour cela, il est nécessaire que la base MongoDB soit en \textbf{\textit{replica set}}.  Un \textbf{\textit{replica set}} est un groupe d’instances MongoDB qui maintiennent les mêmes données. Dans ce groupe, chaque instance est accessible en lecture et il y a une instance primaire qui est la seule à être accessible en écriture. Lorsque des modifications sont faîtes dans cette instance, les modifications sont répliquées aux autres instances du groupe. Dans notre cas, la base de données MongoDB sera l’instance primaire et Elasticsearch, par le biais du plugin, sera considérée comme une instance secondaire. 

Après avoir configurer l'instance de MongoDB en \textbf{\textit{replica set}}, Elasticsearch peut créer des index à partir des données de MongoDB.
Pour effectuer une recherche dans la base à l’aide d’Elasticsearch, nous utiliserons utiliser l’API REST. Cette API utilise le format JSON pour les requêtes et aussi les requêtes en HTTP.


Elle est utilisée de la façon suivante :
\begin{verbatim}
http://host:port/[index]/[type]/[_action|id]
\end{verbatim}
Avec

\begin{itemize}
	\item index : Nom de l’index sur lequel porte l’opération
	\item type : Nom du type de document
	\item \_action : Nom de l’action à effectuer
	\item id : Identifiant du document
\end{itemize}


Par exemple, nous pourrons récupérer le document avec son id mais aussi effectuer une recherche d’un terme particulier :
\begin{verbatim}
http://host:port/[index]/[type]/_search?q=Mai+68
\end{verbatim}

Pour une recherche plus complexe, il faut envoyer une requête au format JSON. Par exemple, si on veut trouver les revues de presse dont le nom contient 'Mai 68'.

\begin{verbatim}
{
    "query": {
        "bool": {
            "must": [
                {
                    "query_string": {
                        "default_field": "Nom",
                        "query": "Mai 68"
                    }
                }
            ],
            "must_not": [ ],
            "should": [ ]
        }
    },
    "from": 0,
    "size": 10,
    "sort": [ ]
}
\end{verbatim}

Avec
\begin{itemize}
	\item La partie \textit{must} contient les termes que doivent avoir les documents recherchés.
	\item La partie \textit{must\_not} les termes que le document ne doit pas contenir
	\item La partie \textit{should} les documents qui auront ces termes seront mis en haut de la liste
	\item \textit{from} correspond à l'indice du document par lequel vous voulez que la liste des résultats commence
	\item \textit {size} est le nombre de résultats que l'on veut récupérer
	\item \textit{sort} permet de trier les résultats
\end{itemize}
Il n'est pas nécessaire de tous les spécifier à chaque requête.


Il est aussi possible d’utiliser le plugin Head d’Elasticsearch. Il permet d’avoir accès à une interface. Depuis cette dernière, nous pouvons consulter l’état d’Elasticsearch et de ses index. Nous accédons aussi à une interface de moteur de recherche. Cependant, ce plugin nous sera moins utile car Elasticsearch sera sur un serveur, nous y accèderons donc principalement à distance.

\subsection{Communication avec php}
\subsubsection{MongoDB}

Pour communiquer avec une base MongoDB, PHP dispose d’une classe Mongo. Celle-ci permet de faire des ajouts, des modifications et des suppressions dans la base de données. 

Dans l'exemple de code suivant, nous ajoutons à la base de données, une nouvelle revue de presse sur Georges Pompidou.

\begin{verbatimtab}[3]
$m = new MongoClient();

// select a database
$db = $m->databaseName;   
$collection = $db->RevueCollection;

$doc = array(
	‘Nom’ => ‘George Pompidou’,
	‘Description’ => ‘Ceci est une revue de presse sur George Pompidou’,
	‘TitreArticles’ => array()
);

$collection->insert($doc);
\end{verbatimtab}


\subsubsection{Elasticsearch}

Pour Elasticsearch, nous utilisons le client PHP officiel pour Elasticsearch. Ainsi, nous pouvons envoyer des requêtes à Elasticsearch et récupérer les résultats de la recherche. Ces requêtes peuvent utiliser la même syntaxe qu’une requête de l’API REST. Par exemple, pour trouver les revues dont la description contient 'Mai 68' :

\begin{verbatimtab}
$json = '{

    "query": {
        "bool": {
            "must": [
                {
                    "query_string": {
                        "default_field": "Description",
                        "query": "Mai 68"
                    }
                }
            ]
        }
    }
}';

$params = [
    'index' => 'RevueIndex',
    'type' => 'Revue',
    'body' => $json
];

$results = $client->search($params);
print_r($results);
\end{verbatimtab}

\newpage
Bien évidemment, Elasticsearch retourne les résultats au format JSON. Mais, le client transforme automatiquement ces données en un format lisible par PHP. Le code suivant correspond aux données de la revue de presse sur Mai 68 dans le format utilisé avec PHP.

\begin{verbatimtab}[3]
Array
(
 [_id] => 1
 [Nom] => Mai 68
 [Description] => Ceci est une revue de presse sur Mai 68
 [TitreArticles] => Array
   (
		[0] => Array
		 (
			[idArticle] => 34354
			[Titre] => Effervescence dans les universités françaises
	   )

		[1] => Array
		 (
			[idArticle] => 78e64a
			[Titre] => Premières barricades de Mai 68
		 )

	)	
)
\end{verbatimtab}
