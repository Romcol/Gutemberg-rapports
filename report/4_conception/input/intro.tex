\section{Introduction}
\label{sec:intro}

Le projet, inscrit dans le cadre de la formation de 4ème année du département Informatique de l’INSA de Rennes et proposé par l’équipe Intuidoc de l’Irisa, a pour but d’améliorer l’accès aux archives de presse locale et ancienne. L’accès aux documents anciens à l’ère du numérique est une problématique réelle. Notre projet concerne donc la mise à disposition de documents de presse ancienne au travers d’une plateforme accessible à tous. C’est à partir d’un prototype, conçu par l’équipe Intuidoc, que nous créerons cette plateforme.
Ce prototype permet d’interpréter des images de journaux anciens (XIXe et XXe) afin de produire une représentation XML du contenu de chaque page. Par ailleurs, Yoann Royer, chef de projet chez Sopra Steria et ancien diplômé de l’INSA, nous offre son accompagnement et ses conseils tout au long du projet.

Le projet se déroule en deux phases : la phase d'analyse et la phase de conception. La première phase qui fut réalisée au premier semestre, consistait à réaliser une pré-étude, une spécification générale, ainsi qu'une planification  du projet. La seconde phase, que nous avons commencé au second semestre concerne la conception et l'implémentation de l'application. Depuis mo-janviern nous ne sommes plus trois étudiants pour terminer la conception de la plateforme et ensuite réaliser son implémentation.

Ce rapport présente la conception logicielle de notre projet qui comprend, entre autre, l'architecture interne de l'application, la modélisation et l'interfaçage des différents modules. Dans un premier temps, nous allons faire un rappel de la phase d'analyse présentant les différentes spécifications fonctionnelles et le choix technologies. Dans un second temps, nous allons présenter l'architecture générale de l'application, ainsi que la modélisation au sein des différents modules utilisés.
