\section{Architecture générale}
\label{sec:generale}
Pour la r�alisation de notre projet, nous allons utiliser des bases de donn�es pour le stockage des donn�es et pour la recherche, un serveur, des framework et une visionneuse de documents. La figure ci-dessous pr�sente l'architecture logicielle g�n�rale de notre projet. Nous allons dans cette partie pr�senter la communication entre les diff�rents modules de cette architecture.
    \begin{figure}[H]
        \centering
        \includegraphics[width=\textwidth]{figures/Archi.png}
            \caption{Architecture globale de la plateforme}
            \label{fig:archi}
    \end{figure}

L'architecture se divise en deux grandes parties : serveur et client. 
Le serveur h�bergeant l'application est un serveur Linux. Il comporte les modules suivants :
\begin{itemize}
	\item Le framework Laravel : il est utilis� pour le codage du centre de la plateforme web en PHP.
	\item La base de donn�es MongoDB : c'est dans cette base que seront stock�es toutes les donn�es de l'application. Laravel pourra donc envoyer des requettes pour r�cup�rer ou modifier des informations.
	\item La base de donn�es ElasticSearch : cette base contiendra exactement les m�mes donn�es que MongoDB et les requ�tes concernant le moteur de recherche y seront envoy�es.
\end{itemize}

Le serveur communique avec le client � travers Laravel via un protocole HTTP. C�t� client nous allons utiliser le framework Bootstrap pour r�aliser une interface graphique adapt�e aux mobiles, tablettes et ordinateurs. Nous utiliserons �galement le plugin Openseadragon qui est une visionneuse pour la consultation de document. 

Nous allons d�tailler dans la suite la mod�lisation de chacun de ces modules de l'architecture g�n�rale du projet, ainsi que la communication entre ceux-ci.
